
\documentclass{article}
%%%%%%%%%%%%%%%%%%%%%%%%%%%%%%%%%%%%%%%%%%%%%%%%%%%%%%%%%%%%%%%%%%%%%%%%%%%%%%%%%%%%%%%%%%%%%%%%%%%%%%%%%%%%%%%%%%%%%%%%%%%%%%%%%%%%%%%%%%%%%%%%%%%%%%%%%%%%%%%%%%%%%%%%%%%%%%%%%%%%%%%%%%%%%%%%%%%%%%%%%%%%%%%%%%%%%%%%%%%%%%%%%%%%%%%%%%%%%%%%%%%%%%%%%%%%
%TCIDATA{OutputFilter=LATEX.DLL}
%TCIDATA{Version=5.50.0.2953}
%TCIDATA{<META NAME="SaveForMode" CONTENT="1">}
%TCIDATA{BibliographyScheme=Manual}
%TCIDATA{Created=Tuesday, July 26, 2022 11:13:04}
%TCIDATA{LastRevised=Tuesday, July 26, 2022 19:48:28}
%TCIDATA{<META NAME="GraphicsSave" CONTENT="32">}
%TCIDATA{<META NAME="DocumentShell" CONTENT="Standard LaTeX\Blank - Standard LaTeX Article">}
%TCIDATA{CSTFile=40 LaTeX article.cst}
%TCIDATA{ComputeDefs=
%$f(x,y)=2x+\frac{2x}{e^{y-x^{1}}}$
%$Y=\left[ 
%\begin{array}{c}
%0 \\ 
%0.177\,71 \\ 
%\allowbreak \allowbreak \allowbreak 0.537\,54 \\ 
%1.\,\allowbreak 033 \\ 
%1.\,\allowbreak 606\,5 \\ 
%\allowbreak \allowbreak 2.\,\allowbreak 224\,6%
%\end{array}%
%\right] $
%$T=\left[ 
%\begin{array}{cccccc}
%1 & 1 & 1 & 1 & 1 & 1 \\ 
%1 & 0.2 & 0.4 & 0.6 & 0.8 & 1.0 \\ 
%1 & 0.04 & 0.16 & 0.36 & 0.64 & 1%
%\end{array}%
%\right] $
%}


\newtheorem{theorem}{Theorem}
\newtheorem{acknowledgement}[theorem]{Acknowledgement}
\newtheorem{algorithm}[theorem]{Algorithm}
\newtheorem{axiom}[theorem]{Axiom}
\newtheorem{case}[theorem]{Case}
\newtheorem{claim}[theorem]{Claim}
\newtheorem{conclusion}[theorem]{Conclusion}
\newtheorem{condition}[theorem]{Condition}
\newtheorem{conjecture}[theorem]{Conjecture}
\newtheorem{corollary}[theorem]{Corollary}
\newtheorem{criterion}[theorem]{Criterion}
\newtheorem{definition}[theorem]{Definition}
\newtheorem{example}[theorem]{Example}
\newtheorem{exercise}[theorem]{Exercise}
\newtheorem{lemma}[theorem]{Lemma}
\newtheorem{notation}[theorem]{Notation}
\newtheorem{problem}[theorem]{Problem}
\newtheorem{proposition}[theorem]{Proposition}
\newtheorem{remark}[theorem]{Remark}
\newtheorem{solution}[theorem]{Solution}
\newtheorem{summary}[theorem]{Summary}
\newenvironment{proof}[1][Proof]{\noindent\textbf{#1.} }{\ \rule{0.5em}{0.5em}}
\input{tcilatex}
\begin{document}


\bigskip $y^{\prime }=2x+\frac{2x}{e^{y-x^{1}}}$

$y(0)=0$

$h=0.2$

$x_{i}=a+ih$

$\bigskip x_{i}=i(0.2)$

$f(x,y)=2x+\frac{2x}{e^{y-x^{1}}}$

$y_{k+1}=y_{k}+hf(x_{k},y_{k})$

$y_{k+1}=0+0.2(f(0,0))\allowbreak =\allowbreak 0.0$

$y_{k+1}=0+0.2(f(0.2,0))=\allowbreak 0.177\,71$

$y_{k+1}=0.177\,71+0.2(f(0.4,\allowbreak 0.177\,71))=\allowbreak 0.537\,54$

$y_{k+1}=\allowbreak \allowbreak 0.537\,54+0.2(f(0.6,\allowbreak \allowbreak
\allowbreak 0.537\,54))=\allowbreak 1.\,\allowbreak 033$

$y_{k+1}=\allowbreak 1.\,\allowbreak 033+0.2(f(0.8,\allowbreak
1.\,\allowbreak 033))=\allowbreak 1.\,\allowbreak 606\,5$

$y_{k+1}=\allowbreak \allowbreak 1.\,\allowbreak 606\,5+0.2(f(1,\allowbreak
\allowbreak \allowbreak 1.\,\allowbreak 606\,5))=\allowbreak 2.\,\allowbreak
224\,6$

\bigskip $%
\begin{array}{cc}
x_{k} & y_{k} \\ 
0 & 0 \\ 
0.2 & \allowbreak \allowbreak 0.177\,71 \\ 
0.4 & \allowbreak \allowbreak \allowbreak 0.537\,54 \\ 
0.6 & \allowbreak 1.\,\allowbreak 033 \\ 
0.8 & \allowbreak 1.\,\allowbreak 606\,5\allowbreak  \\ 
1 & \allowbreak \allowbreak 2.\,\allowbreak 224\,6%
\end{array}%
\left[ 
\begin{array}{c}
0 \\ 
0.177\,71 \\ 
\allowbreak \allowbreak \allowbreak 0.537\,54 \\ 
1.\,\allowbreak 033 \\ 
1.\,\allowbreak 606\,5 \\ 
\allowbreak \allowbreak 2.\,\allowbreak 224\,6%
\end{array}%
\right] $\FRAME{dtbpFX}{4.4996in}{3in}{0pt}{}{}{Plot}{\special{language
"Scientific Word";type "MAPLEPLOT";width 4.4996in;height 3in;depth
0pt;display "USEDEF";plot_snapshots TRUE;mustRecompute FALSE;lastEngine
"MuPAD";xmin "-5";xmax "5";xviewmin "-0.00010000010002";xviewmax
"1.00010000010002";yviewmin "-0.846497079213021";yviewmax
"2.22490707900002";plottype 4;axesFont "Times New
Roman,12,0000000000,useDefault,normal";numpoints 100;plotstyle
"patchnogrid";axesstyle "normal";axestips FALSE;xis \TEXUX{x};var1name
\TEXUX{$x$};function
\TEXUX{$\MATRIX{2,6}{c}\VR{,,c,,,}{,,c,,,}{,,,,,}\HR{,,,,,,}\CELL{0}\CELL{0}%
\CELL{0.2}\CELL{\allowbreak \allowbreak
0.177\,71}\CELL{0.4}\CELL{\allowbreak \allowbreak \allowbreak
0.537\,54}\CELL{0.6}\CELL{\allowbreak 1.\,\allowbreak
033}\CELL{0.8}\CELL{\allowbreak 1.\,\allowbreak 606\,5\allowbreak
}\CELL{1}\CELL{\allowbreak \allowbreak 2.\,\allowbreak 224\,6}$};linecolor
"blue";linestyle 1;pointplot TRUE;pointstyle "point";linethickness
1;lineAttributes "Solid";curveColor "[flat::RGB:0x000000ff]";curveStyle
"Point";function \TEXUX{$-0.846\,19+5.\,\allowbreak 198\,4x-3.\,\allowbreak
167\,8x^{2}$};linecolor "black";linestyle 1;discont FALSE;pointstyle
"point";linethickness 1;lineAttributes "Solid";var1range
"0,1";num-x-gridlines 100;curveColor "[flat::RGB:0000000000]";curveStyle
"Line";discont FALSE;rangeset"X";VCamFile 'RFNFEL0A.xvz';valid_file
"T";tempfilename 'RFNFCK03.wmf';tempfile-properties "XPR";}}

$f_{1}(x)=1$

$f_{2}(x)=x$

$f_{3}(x)=x^{2}$

$A=\left[ 
\begin{array}{ccc}
1 & 1 & 1 \\ 
1 & 0.2 & 0.2^{2} \\ 
1 & 0.4 & 0.4^{2} \\ 
1 & 0.6 & 0.6^{2} \\ 
1 & 0.8 & 0.8^{2} \\ 
1 & 1.0 & 1%
\end{array}%
\right] $, transpose: $\left[ 
\begin{array}{cccccc}
1 & 1 & 1 & 1 & 1 & 1 \\ 
1 & 0.2 & 0.4 & 0.6 & 0.8 & 1.0 \\ 
1 & 0.04 & 0.16 & 0.36 & 0.64 & 1%
\end{array}%
\right] \allowbreak $

Columna: Valores de Y en S:

$Y=\left[ 
\begin{array}{c}
0 \\ 
0.177\,71 \\ 
\allowbreak \allowbreak \allowbreak 0.537\,54 \\ 
1.\,\allowbreak 033 \\ 
1.\,\allowbreak 606\,5 \\ 
\allowbreak \allowbreak 2.\,\allowbreak 224\,6%
\end{array}%
\right] $

$T=\left[ 
\begin{array}{cccccc}
1 & 1 & 1 & 1 & 1 & 1 \\ 
1 & 0.2 & 0.4 & 0.6 & 0.8 & 1.0 \\ 
1 & 0.04 & 0.16 & 0.36 & 0.64 & 1%
\end{array}%
\right] \allowbreak $

$C=\left[ 
\begin{array}{cccccc}
1 & 1 & 1 & 1 & 1 & 1 \\ 
1 & 0.2 & 0.4 & 0.6 & 0.8 & 1.0 \\ 
1 & 0.04 & 0.16 & 0.36 & 0.64 & 1%
\end{array}%
\right] \allowbreak \left[ 
\begin{array}{ccc}
1 & 1 & 1 \\ 
1 & 0.2 & 0.2^{2} \\ 
1 & 0.4 & 0.4^{2} \\ 
1 & 0.6 & 0.6^{2} \\ 
1 & 0.8 & 0.8^{2} \\ 
1 & 1.0 & 1%
\end{array}%
\right] =\allowbreak \left[ 
\begin{array}{ccc}
6.0 & 4.0 & 3.\,\allowbreak 2 \\ 
4.0 & 3.\,\allowbreak 2 & 2.\,\allowbreak 8 \\ 
3.\,\allowbreak 2 & 2.\,\allowbreak 8 & 2.\,\allowbreak 566\,4%
\end{array}%
\right] \allowbreak $

$D=\left[ 
\begin{array}{cccccc}
1 & 1 & 1 & 1 & 1 & 1 \\ 
1 & 0.2 & 0.4 & 0.6 & 0.8 & 1.0 \\ 
1 & 0.04 & 0.16 & 0.36 & 0.64 & 1%
\end{array}%
\right] \left[ 
\begin{array}{c}
0 \\ 
0.177\,71 \\ 
\allowbreak \allowbreak \allowbreak 0.537\,54 \\ 
1.\,\allowbreak 033 \\ 
1.\,\allowbreak 606\,5 \\ 
\allowbreak \allowbreak 2.\,\allowbreak 224\,6%
\end{array}%
\right] =\allowbreak \left[ 
\begin{array}{c}
5.\,\allowbreak 579\,4 \\ 
4.\,\allowbreak 380\,2 \\ 
3.\,\allowbreak 717\,8%
\end{array}%
\right] \allowbreak $

$F=\left[ 
\begin{array}{ccc}
6.0 & 4.0 & 3.\,\allowbreak 2 \\ 
4.0 & 3.\,\allowbreak 2 & 2.\,\allowbreak 8 \\ 
3.\,\allowbreak 2 & 2.\,\allowbreak 8 & 2.\,\allowbreak 566\,4%
\end{array}%
\right] ^{-1}\left[ 
\begin{array}{c}
5.\,\allowbreak 579\,4 \\ 
4.\,\allowbreak 380\,2 \\ 
3.\,\allowbreak 717\,8%
\end{array}%
\right] \allowbreak =\allowbreak \left[ 
\begin{array}{c}
-0.846\,19 \\ 
5.\,\allowbreak 198\,4 \\ 
-3.\,\allowbreak 167\,8%
\end{array}%
\right] \allowbreak $

\[
g(x)=-0.846\,19+5.\,\allowbreak 198\,4x-3.\,\allowbreak 167\,8x^{2}
\]

\bigskip 

\bigskip 

\bigskip 

\bigskip 

\bigskip 

\bigskip 

\bigskip 

\bigskip 

\bigskip 

\bigskip 

\bigskip 1- Columna 1: Reemplazamos cada valor de x con A=\{1\}

2- Columna 2: Reemplazamos cada valor de x con A=\{x\}

3- Columna 3: Reemplazamos cada valor de x con A=\{x2\}

$A=\left[ 
\begin{array}{ccc}
1 & 1 & 1 \\ 
1 & 1.2 & 1.44 \\ 
1 & 1.4 & 1.96 \\ 
1 & 1.6 & 2.56 \\ 
1 & 1.8 & 3.24 \\ 
1 & 2.0 & 4%
\end{array}%
\right] $

\bigskip Columna: Valores de Y en S:

$Y=\left[ 
\begin{array}{c}
1 \\ 
1 \\ 
1.\,\allowbreak 016\,4 \\ 
1.\,\allowbreak 041\,8 \\ 
1.\,\allowbreak 071\,3 \\ 
1.\,\allowbreak 102\,2%
\end{array}%
\right] $

\bigskip

Calculamos la transpuesta de A:

$A$, transpose: $\left[ 
\begin{array}{cccccc}
1 & 1 & 1 & 1 & 1 & 1 \\ 
1 & 1.\,\allowbreak 2 & 1.\,\allowbreak 4 & 1.\,\allowbreak 6 & 
1.\,\allowbreak 8 & 2.0 \\ 
1 & 1.\,\allowbreak 44 & 1.\,\allowbreak 96 & 2.\,\allowbreak 56 & 
3.\,\allowbreak 24 & 4%
\end{array}%
\right] \allowbreak $

$A^{t}=\left[ 
\begin{array}{cccccc}
1 & 1 & 1 & 1 & 1 & 1 \\ 
1 & 1.\,\allowbreak 2 & 1.\,\allowbreak 4 & 1.\,\allowbreak 6 & 
1.\,\allowbreak 8 & 2.0 \\ 
1 & 1.\,\allowbreak 44 & 1.\,\allowbreak 96 & 2.\,\allowbreak 56 & 
3.\,\allowbreak 24 & 4%
\end{array}%
\right] $

!!Cuidado con el orden de multiplicar

Calculamos el producto de A(transpuesta) x A:

$C=\left[ 
\begin{array}{cccccc}
1 & 1 & 1 & 1 & 1 & 1 \\ 
1 & 1.\,\allowbreak 2 & 1.\,\allowbreak 4 & 1.\,\allowbreak 6 & 
1.\,\allowbreak 8 & 2.0 \\ 
1 & 1.\,\allowbreak 44 & 1.\,\allowbreak 96 & 2.\,\allowbreak 56 & 
3.\,\allowbreak 24 & 4%
\end{array}%
\right] \left[ 
\begin{array}{ccc}
1 & 1 & 1 \\ 
1 & 1.2 & 1.44 \\ 
1 & 1.4 & 1.96 \\ 
1 & 1.6 & 2.56 \\ 
1 & 1.8 & 3.24 \\ 
1 & 2.0 & 4%
\end{array}%
\right] =\allowbreak \left[ 
\begin{array}{ccc}
6.0 & 9.0 & 14.\,\allowbreak 2 \\ 
9.0 & 14.\,\allowbreak 2 & 23.\,\allowbreak 4 \\ 
14.\,\allowbreak 2 & 23.\,\allowbreak 4 & 39.\,\allowbreak 966%
\end{array}%
\right] \allowbreak $

\bigskip

Calculamos el producto de A(transpuesta) x Y:

$D=\left[ 
\begin{array}{cccccc}
1 & 1 & 1 & 1 & 1 & 1 \\ 
1 & 1.\,\allowbreak 2 & 1.\,\allowbreak 4 & 1.\,\allowbreak 6 & 
1.\,\allowbreak 8 & 2.0 \\ 
1 & 1.\,\allowbreak 44 & 1.\,\allowbreak 96 & 2.\,\allowbreak 56 & 
3.\,\allowbreak 24 & 4%
\end{array}%
\right] \left[ 
\begin{array}{c}
1 \\ 
1 \\ 
1.\,\allowbreak 016\,4 \\ 
1.\,\allowbreak 041\,8 \\ 
1.\,\allowbreak 071\,3 \\ 
1.\,\allowbreak 102\,2%
\end{array}%
\right] =\allowbreak \left[ 
\begin{array}{c}
6.\,\allowbreak 231\,7 \\ 
9.\,\allowbreak 422\,6 \\ 
14.\,\allowbreak 979%
\end{array}%
\right] $

\bigskip

Multiplicamos C$^{-1}$ x D:

\bigskip $R=\left[ 
\begin{array}{ccc}
6.0 & 9.0 & 14.\,\allowbreak 2 \\ 
9.0 & 14.\,\allowbreak 2 & 23.\,\allowbreak 4 \\ 
14.\,\allowbreak 2 & 23.\,\allowbreak 4 & 39.\,\allowbreak 966%
\end{array}%
\right] ^{-1}\left[ 
\begin{array}{c}
6.\,\allowbreak 231\,7 \\ 
9.\,\allowbreak 422\,6 \\ 
14.\,\allowbreak 979%
\end{array}%
\right] =\allowbreak \left[ 
\begin{array}{c}
1.\,\allowbreak 075\,3 \\ 
-0.170\,54 \\ 
9.\,\allowbreak 258\,4\times 10^{-2}%
\end{array}%
\right] \allowbreak 
\begin{array}{c}
a0 \\ 
a1 \\ 
a2%
\end{array}%
$

\bigskip Reemplazamos R en la f(x) con los a sub n.

$A=\{1,x,x^{2}\}$

$f(x)=a_{0}+a_{1}x+a_{2}x^{2}$

\[
f(x)=1.0753-0.17054x+9.\,\allowbreak 258\,4\times 10^{-2}x^{2}
\]

\bigskip

\bigskip

\bigskip

\bigskip

$\allowbreak $

\bigskip

Calculamos el producto de A(transpuesta) x Y:

$D=\left[ 
\begin{array}{cccccc}
1 & 1 & 1 & 1 & 1 & 1 \\ 
1 & 1.\,\allowbreak 2 & 1.\,\allowbreak 4 & 1.\,\allowbreak 6 & 
1.\,\allowbreak 8 & 2.0 \\ 
1 & 1.\,\allowbreak 44 & 1.\,\allowbreak 96 & 2.\,\allowbreak 56 & 
3.\,\allowbreak 24 & 4%
\end{array}%
\right] \left[ 
\begin{array}{c}
1 \\ 
1 \\ 
1.\,\allowbreak 016\,4 \\ 
1.\,\allowbreak 041\,8 \\ 
1.\,\allowbreak 071\,3 \\ 
1.\,\allowbreak 102\,2%
\end{array}%
\right] =\allowbreak \left[ 
\begin{array}{c}
6.\,\allowbreak 231\,7 \\ 
9.\,\allowbreak 422\,6 \\ 
14.\,\allowbreak 979%
\end{array}%
\right] $

\bigskip

Multiplicamos C$^{-1}$ x D:

\bigskip $R=\left[ 
\begin{array}{ccc}
6.0 & 9.0 & 14.\,\allowbreak 2 \\ 
9.0 & 14.\,\allowbreak 2 & 23.\,\allowbreak 4 \\ 
14.\,\allowbreak 2 & 23.\,\allowbreak 4 & 39.\,\allowbreak 966%
\end{array}%
\right] ^{-1}\left[ 
\begin{array}{c}
6.\,\allowbreak 231\,7 \\ 
9.\,\allowbreak 422\,6 \\ 
14.\,\allowbreak 979%
\end{array}%
\right] =\allowbreak \left[ 
\begin{array}{c}
1.\,\allowbreak 075\,3 \\ 
-0.170\,54 \\ 
9.\,\allowbreak 258\,4\times 10^{-2}%
\end{array}%
\right] \allowbreak 
\begin{array}{c}
a0 \\ 
a1 \\ 
a2%
\end{array}%
$

\bigskip Reemplazamos R en la f(x) con los a sub n.

$A=\{1,x,x^{2}\}$

$f(x)=a_{0}+a_{1}x+a_{2}x^{2}$

\[
f(x)=1.0753-0.17054x+9.\,\allowbreak 258\,4\times 10^{-2}x^{2} 
\]

\end{document}
