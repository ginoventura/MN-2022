
\documentclass{article}
%%%%%%%%%%%%%%%%%%%%%%%%%%%%%%%%%%%%%%%%%%%%%%%%%%%%%%%%%%%%%%%%%%%%%%%%%%%%%%%%%%%%%%%%%%%%%%%%%%%%%%%%%%%%%%%%%%%%%%%%%%%%%%%%%%%%%%%%%%%%%%%%%%%%%%%%%%%%%%%%%%%%%%%%%%%%%%%%%%%%%%%%%%%%%%%%%%%%%%%%%%%%%%%%%%%%%%%%%%%%%%%%%%%%%%%%%%%%%%%%%%%%%%%%%%%%
\usepackage{amsmath}

\setcounter{MaxMatrixCols}{10}
%TCIDATA{OutputFilter=LATEX.DLL}
%TCIDATA{Version=5.50.0.2953}
%TCIDATA{<META NAME="SaveForMode" CONTENT="1">}
%TCIDATA{BibliographyScheme=Manual}
%TCIDATA{Created=Tuesday, July 12, 2022 18:24:21}
%TCIDATA{LastRevised=Tuesday, July 26, 2022 09:41:59}
%TCIDATA{<META NAME="GraphicsSave" CONTENT="32">}
%TCIDATA{<META NAME="DocumentShell" CONTENT="Standard LaTeX\Blank - Standard LaTeX Article">}
%TCIDATA{CSTFile=40 LaTeX article.cst}

\newtheorem{theorem}{Theorem}
\newtheorem{acknowledgement}[theorem]{Acknowledgement}
\newtheorem{algorithm}[theorem]{Algorithm}
\newtheorem{axiom}[theorem]{Axiom}
\newtheorem{case}[theorem]{Case}
\newtheorem{claim}[theorem]{Claim}
\newtheorem{conclusion}[theorem]{Conclusion}
\newtheorem{condition}[theorem]{Condition}
\newtheorem{conjecture}[theorem]{Conjecture}
\newtheorem{corollary}[theorem]{Corollary}
\newtheorem{criterion}[theorem]{Criterion}
\newtheorem{definition}[theorem]{Definition}
\newtheorem{example}[theorem]{Example}
\newtheorem{exercise}[theorem]{Exercise}
\newtheorem{lemma}[theorem]{Lemma}
\newtheorem{notation}[theorem]{Notation}
\newtheorem{problem}[theorem]{Problem}
\newtheorem{proposition}[theorem]{Proposition}
\newtheorem{remark}[theorem]{Remark}
\newtheorem{solution}[theorem]{Solution}
\newtheorem{summary}[theorem]{Summary}
\newenvironment{proof}[1][Proof]{\noindent\textbf{#1.} }{\ \rule{0.5em}{0.5em}}
\input{tcilatex}
\begin{document}


\begin{equation*}
\text{\textit{M\'{e}todo de Newton-Raphson}} 
\end{equation*}

\begin{equation*}
\mathbf{Definicion} 
\end{equation*}

\textit{Dada la sucesi\'{o}n} $\left\{ P_{n}\right\} $ \textit{que converge
al punto} $p$ \textit{diremos que la misma tiene convergencia de orden} $%
\alpha $ $(\alpha >0)$ \textit{si existe }$\beta $ $\in R^{+}$ \textit{tal
que} $\lim_{n\rightarrow \infty }\frac{\left\vert P_{n+1}-P\right\vert }{%
\left\vert P_{n}-P\right\vert ^{\alpha }}=\beta $

\begin{equation*}
\mathbf{Teorema} 
\end{equation*}

\textit{Sea} $g:R\rightarrow R$ \textit{de clase} $C^{\infty }$ \textit{y
sea la sucesion} $\left\{ P_{n}\right\} $ \textit{dada por} $%
P_{n}=g(P_{n-1}) $\textit{. Si} $p$ \textit{es un punto fijo de }$g$ \textit{%
y} $g^{(k)}(p)=0$ \textit{para} $k=1,2,3,4,\cdot \cdot \cdot ,m-1$ \textit{y}
$g^{(m)}(p)\neq 0 $, \textit{entonces la sucesi\'{o}n} $\left\{
P_{n}\right\} $ \textit{converge a }$P$ \textit{con orden de convergencia }$m
$ \textit{y constante asint\'{o}tica:}

\begin{equation*}
\beta =\frac{\left\vert g^{(m)}(p)\right\vert }{m!} 
\end{equation*}

\begin{equation*}
\mathbf{Demostracion} 
\end{equation*}

\textit{Por un resultado anterior }$\lim_{n\rightarrow \infty }P_{n}=P$ 
\textit{y }$g(P)$. \textit{Usando ahora el desarrollo de Taylor de }$g$ 
\textit{al rededor de }$P$\textit{\ tenemos: }$g(x)=\dsum\limits_{k=0}^{m-1}%
\frac{g^{(k)}(P)}{k!}(x-P)^{k}+R_{m-1}(x)=g(P)+\dsum\limits_{k=1}^{m-1}\frac{%
g^{(k)}(P)}{k!}+\frac{g^{(m)}(\xi )}{m!}(x-P)^{m}=P+\frac{g^{(m)}(\xi )}{m!}%
(x-P)^{m}$, \textit{de donde }$g(x-P)=\frac{g^{(m)}(\xi )}{m!}(x-P)^{m}$

\textit{Valuando en }$x=P_{n}$ \textit{tenemos }$P_{n+1}-P=\frac{g^{(m)}(\xi
)}{m!}(P_{n}-P)^{m}$ \textit{tomando valor absoluto se tiene }$\left\vert
P_{n+1}-P\right\vert =\frac{g^{(m)}(\xi _{n})}{m!}\left\vert
P_{n}-P\right\vert ^{m}$

\textit{Equivalentemente }$\frac{\left\vert P_{n+1}-P\right\vert }{%
\left\vert P_{n}-P\right\vert ^{m}}=\frac{g^{(m)}(\xi _{n})}{m!}\mathit{\ }$%
\textit{tomando limite }$n\rightarrow \infty $ \textit{obtenemos que: }%
\begin{equation*}
\underset{n\rightarrow \infty }{\lim }\frac{\left\vert P_{n+1}-P\right\vert 
}{\left\vert P_{n}-P\right\vert ^{m}}=\frac{g^{(m)}(\xi _{n})}{m!}
\end{equation*}%
\emph{Corolario: El metodo de Newton-Raphson tiene convergencia cuadr\'{a}%
tica.}

\textit{Demostraci\'{o}n: Sea }$g(x)=x-\frac{f(x)}{f^{\prime }(x)}$\bigskip\ 
\textit{y tal que }$g(P)=P$, \textit{entonces }$f(P)=0$

$\bullet $ \textit{Debemos ver que }$g^{\prime }(P)=0$ \textit{y }$g^{\prime
\prime }(P)\neq 0.$

$g^{\prime }(x)=1-\frac{\left[ f^{\prime }(x)\right] ^{2}-f(x)f^{\prime
\prime }(x)}{\left[ f^{\prime }(x)\right] ^{2}}=\frac{f(x)f^{\prime \prime
}(x)}{\left[ f^{\prime }(x)\right] ^{2}}\rightarrow g^{\prime }(P)=0$ 
\textit{ya que }$f(P)=0$.

$\bullet $ \textit{Ahora la derivada segunda:}

$g^{\prime \prime }(x)=f^{\prime }(x)\frac{f^{\prime \prime }(x)}{\left[
f^{\prime }(x)\right] ^{2}}+f(x)\left[ \frac{\partial f}{\partial x}(\frac{%
f^{\prime \prime }(x)}{\left[ f^{\prime }(x)\right] ^{2}})\right] =g^{\prime
\prime }(P)=\frac{f^{\prime \prime }(P)}{f^{\prime }(P)}\neq 0$. \textit{ya
que }$P\mathit{\ }$\textit{no es ra\'{\i}z m\'{u}ltiple.}

\end{document}
