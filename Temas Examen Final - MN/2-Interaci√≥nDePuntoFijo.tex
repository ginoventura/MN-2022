
\documentclass{article}
%%%%%%%%%%%%%%%%%%%%%%%%%%%%%%%%%%%%%%%%%%%%%%%%%%%%%%%%%%%%%%%%%%%%%%%%%%%%%%%%%%%%%%%%%%%%%%%%%%%%%%%%%%%%%%%%%%%%%%%%%%%%%%%%%%%%%%%%%%%%%%%%%%%%%%%%%%%%%%%%%%%%%%%%%%%%%%%%%%%%%%%%%%%%%%%%%%%%%%%%%%%%%%%%%%%%%%%%%%%%%%%%%%%%%%%%%%%%%%%%%%%%%%%%%%%%
\usepackage{amsmath}

\setcounter{MaxMatrixCols}{10}
%TCIDATA{OutputFilter=LATEX.DLL}
%TCIDATA{Version=5.50.0.2953}
%TCIDATA{<META NAME="SaveForMode" CONTENT="1">}
%TCIDATA{BibliographyScheme=Manual}
%TCIDATA{Created=Wednesday, July 13, 2022 12:59:48}
%TCIDATA{LastRevised=Tuesday, July 26, 2022 09:45:53}
%TCIDATA{<META NAME="GraphicsSave" CONTENT="32">}
%TCIDATA{<META NAME="DocumentShell" CONTENT="Standard LaTeX\Blank - Standard LaTeX Article">}
%TCIDATA{CSTFile=40 LaTeX article.cst}

\newtheorem{theorem}{Theorem}
\newtheorem{acknowledgement}[theorem]{Acknowledgement}
\newtheorem{algorithm}[theorem]{Algorithm}
\newtheorem{axiom}[theorem]{Axiom}
\newtheorem{case}[theorem]{Case}
\newtheorem{claim}[theorem]{Claim}
\newtheorem{conclusion}[theorem]{Conclusion}
\newtheorem{condition}[theorem]{Condition}
\newtheorem{conjecture}[theorem]{Conjecture}
\newtheorem{corollary}[theorem]{Corollary}
\newtheorem{criterion}[theorem]{Criterion}
\newtheorem{definition}[theorem]{Definition}
\newtheorem{example}[theorem]{Example}
\newtheorem{exercise}[theorem]{Exercise}
\newtheorem{lemma}[theorem]{Lemma}
\newtheorem{notation}[theorem]{Notation}
\newtheorem{problem}[theorem]{Problem}
\newtheorem{proposition}[theorem]{Proposition}
\newtheorem{remark}[theorem]{Remark}
\newtheorem{solution}[theorem]{Solution}
\newtheorem{summary}[theorem]{Summary}
\newenvironment{proof}[1][Proof]{\noindent\textbf{#1.} }{\ \rule{0.5em}{0.5em}}
\input{tcilatex}
\begin{document}


\begin{equation*}
\text{\emph{Iteraci\'{o}n de Punto Fijo}}
\end{equation*}

\textit{Sea }$f:R\rightarrow R$ \textit{continua y sea }$\left( a,b\right)
\subset dom(f)$ \textit{tal que} $f(a)f(b)<0$ $\mathit{\ }$\textit{entonces
existe un punto }$P$ \textit{tal que }$f(P)=0$.

\textit{El problema a resolver es obtener }$P$,\textit{\ esto es, hallar una
solucion de la ecuaci\'{o}n }$f(x)=0$.

\textit{Si }$f(x)=0$,\textit{\ entonces,}$f(x)=0\leftrightarrow f(x)+x=x$ 
\textit{y llamando }$g(x)=f(x)+x$ \textit{tenemos que }$g(x)=x.$

\begin{equation*}
\text{\emph{Definici\'{o}n}} 
\end{equation*}

\bigskip \textit{Sea }$g:R\rightarrow R$. \textit{Si }$x$\textit{\ }$\in $ $%
dom(g)$ \textit{tal que }$g(x)=x$\textit{\ se dice que }$x$\textit{\ es un
punto fijo de }$g.$

\begin{equation*}
\text{\emph{Teorema}} 
\end{equation*}

\textit{Sea }$g:\left[ a,b\right] \rightarrow \left[ a,b\right] $ \fbox{$%
a\leq x\leq b\rightarrow a\leq g(x)\leq b$} \textit{continua, entonces} $g$%
\textit{\ tiene un punto fijo en el intervalo }$\left[ a,b\right] $.

\begin{equation*}
\text{\emph{Demostraci\'{o}n}} 
\end{equation*}

\textit{Sea }$h:\left[ a,b\right] \rightarrow R$ \textit{dada por }$%
h(x)=x-g(x)$,\textit{\ entonces }$h$ \textit{es continua. Por otro lado }$%
h(a)=a-g(a)\leq 0$ \textit{y }$h(b)=b-g(b)\geq 0$. \textit{Luego, por el
teorema de los valores intermedios, existe }$p\in \left[ a,b\right] $ 
\textit{tal que }$h(p)=0$ \textit{\'{o} equivalentemente }$g(p)=p$.

\FRAME{dtbpFX}{3.0969in}{2.0652in}{0pt}{}{}{Plot}{\special{language
"Scientific Word";type "MAPLEPLOT";width 3.0969in;height 2.0652in;depth
0pt;display "USEDEF";plot_snapshots TRUE;mustRecompute FALSE;lastEngine
"MuPAD";xmin "-1";xmax "1";xviewmin "-1.00020000020004";xviewmax
"1.00020000020004";yviewmin "-4.00080000080016";yviewmax
"4.00080000080016";plottype 4;axesFont "Times New
Roman,12,0000000000,useDefault,normal";numpoints 100;plotstyle
"patch";axesstyle "normal";axestips FALSE;xis \TEXUX{x};var1name
\TEXUX{$x$};function \TEXUX{$4x$};linecolor "black";linestyle 1;pointstyle
"point";linethickness 1;lineAttributes "Solid";var1range
"-1,1";num-x-gridlines 100;curveColor "[flat::RGB:0000000000]";curveStyle
"Line";rangeset"X";function \TEXUX{$2x^{3}+1$};linecolor "blue";linestyle
1;pointstyle "point";linethickness 1;lineAttributes "Solid";var1range
"-1,1";num-x-gridlines 100;curveColor "[flat::RGB:0x000000ff]";curveStyle
"Line";rangeset"X";function
\TEXUX{$\MATRIX{2,1}{c}\VR{,,c,,,}{,,c,,,}{,,,,,}\HR{,}\CELL{0.26}%
\CELL{1}$};linecolor "green";linestyle 1;pointplot TRUE;pointstyle
"point";linethickness 1;lineAttributes "Solid";curveColor
"[flat::RGB:0x00008000]";curveStyle "Point";VCamFile
'REYWQ70A.xvz';valid_file "T";tempfilename
'REYWJ400.wmf';tempfile-properties "XPR";}}

\begin{equation*}
\text{\emph{Teorema}} 
\end{equation*}

\textit{Sea }$g:\left[ a,b\right] \rightarrow R$ \textit{de clase }$C^{1}$ 
\textit{tal que }$\left\vert g^{\prime }(x)\right\vert \leq r<1$\textit{\
para todo }$x\in \left[ a,b\right] $. \textit{Si }$g$ \textit{tiene punto
fijo en }$\left[ a,b\right] $\textit{\ \'{e}ste es \'{u}nico.}

\begin{equation*}
\text{\emph{Demostraci\'{o}n}} 
\end{equation*}

\textit{Sean }$p$ \textit{y }$q$ \textit{puntos fijos de }$g$, \textit{%
entonces }$\left\vert p-q\right\vert =\left\vert g(p)-g(q)\right\vert $ 
\textit{por teorema de valor medio }$\left\vert g(p)-g(q)\right\vert
=\left\vert g^{\prime }(\xi )(p-q)\right\vert =\left\vert g^{\prime }(\xi
)\right\vert \left\vert (p-q)\right\vert \leq r\left\vert p-q\right\vert $ 
\textit{si }$\left\vert p\neq q\right\vert >0$\textit{\ y por lo tanto la
desigualdad anterior equivalente a }$1\leq r$. \textit{Lo que contradice que 
}$1<r$ \textit{por lo tanto la suposici\'{o}n }$p\neq q$ \textit{es falsa,
luego }$p=q$.

$\bullet $ \textit{Sea }$g:\left[ a,b\right] \rightarrow R$ \textit{de clase 
}$C^{1}$ \textit{y }$\left\vert g^{\prime }(x)\right\vert \leq r<1$ \textit{%
para todo }$x\in \left[ a,b\right] $\textit{\ consideremos la sucesi\'{o}n }$%
\left\{ P_{n}\right\} $ \textit{dada por }$P_{n+1}=g(P_{n})$ \textit{con }$%
P_{0}\in \left[ a,b\right] $. \textit{Veamos que }$\left\{ P_{n}\right\} $%
\textit{\ converge al \'{u}nico punto fijo de }$g$.

$\left\vert P_{2}-P_{1}\right\vert =\left\vert g(P_{1})-g(P_{0})\right\vert
\leq r\left\vert P_{1}-P_{0}\right\vert $, $\left\vert
P_{3}-P_{2}\right\vert =\left\vert g(P_{2})-g(P_{1})\right\vert \leq
r\left\vert P_{2}-P_{1}\right\vert \leq r^{2}\left\vert
P_{1}-P_{0}\right\vert $ \textit{y en general }$\fbox{$\left\vert
P_{n+1}-P_{n}\right\vert \leq r^{n}\left\vert P_{1}-P_{0}\right\vert $}$

\textit{Sean ahora }$n,m\in Z^{+}$ \textit{con }$n>m$ \textit{entonces: }$%
\left\vert P_{n}-P_{m}\right\vert =\left\vert
(P_{n}-P_{n-1})+(P_{n-1}-P_{n-2})+P_{n-2}+\cdots +(P_{m+1}-P_{m})\right\vert
\leq \sum\limits_{k=m+1}^{n}\left\vert P_{k}-P_{k-1}\right\vert \leq
\sum\limits_{k=m+1}^{n}r^{k-1}\left\vert P_{1}-P_{0}\right\vert =\left\vert
P_{1}-P_{0}\right\vert \sum\limits_{k=m+1}^{n}r^{k-1}=\left\vert
P_{1}-P_{0}\right\vert \sum\limits_{k=m}^{n-1}r^{k}=\left\vert
P_{1}-P_{0}\right\vert \frac{r^{m}-r^{n}}{1-r}$ \textit{tiende a 0 cuando }$%
m,n\rightarrow \infty $ \textit{\ por lo tanto }$\left\vert
P_{n}-P_{m}\right\vert \rightarrow 0$ \textit{cuando }$m,n\rightarrow \infty 
$. \textit{Por lo tanto converge, esto es el }$\fbox{$\underset{n\rightarrow
\infty }{\mathbf{\lim }}P_{n}=P$}$

\textit{En consecuencia, }$P$ \textit{es el }$\underset{n\rightarrow \infty }%
{\lim }g(P_{n-1})=g(\underset{n\rightarrow \infty }{\lim }P_{n-1})=P$

\end{document}
