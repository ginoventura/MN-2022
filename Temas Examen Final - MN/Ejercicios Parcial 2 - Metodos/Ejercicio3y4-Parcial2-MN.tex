
\documentclass{article}
%%%%%%%%%%%%%%%%%%%%%%%%%%%%%%%%%%%%%%%%%%%%%%%%%%%%%%%%%%%%%%%%%%%%%%%%%%%%%%%%%%%%%%%%%%%%%%%%%%%%%%%%%%%%%%%%%%%%%%%%%%%%%%%%%%%%%%%%%%%%%%%%%%%%%%%%%%%%%%%%%%%%%%%%%%%%%%%%%%%%%%%%%%%%%%%%%%%%%%%%%%%%%%%%%%%%%%%%%%%%%%%%%%%%%%%%%%%%%%%%%%%%%%%%%%%%
%TCIDATA{OutputFilter=LATEX.DLL}
%TCIDATA{Version=5.50.0.2953}
%TCIDATA{<META NAME="SaveForMode" CONTENT="1">}
%TCIDATA{BibliographyScheme=Manual}
%TCIDATA{Created=Monday, June 13, 2022 20:17:50}
%TCIDATA{LastRevised=Tuesday, July 26, 2022 11:07:17}
%TCIDATA{<META NAME="GraphicsSave" CONTENT="32">}
%TCIDATA{<META NAME="DocumentShell" CONTENT="Standard LaTeX\Blank - Standard LaTeX Article">}
%TCIDATA{CSTFile=40 LaTeX article.cst}
%TCIDATA{ComputeDefs=
%$f(x,y)=\frac{x-y}{1+y^{4}}$
%$A=\left[ 
%\begin{array}{ccc}
%1 & 0 & 0 \\ 
%1 & 0.33334 & 0,11115556 \\ 
%1 & 0.50001 & 0.25 \\ 
%1 & 0.66668 & 0.44 \\ 
%1 & 0.83335 & 0.694 \\ 
%1 & 1 & 1%
%\end{array}%
%\right] $
%}


\newtheorem{theorem}{Theorem}
\newtheorem{acknowledgement}[theorem]{Acknowledgement}
\newtheorem{algorithm}[theorem]{Algorithm}
\newtheorem{axiom}[theorem]{Axiom}
\newtheorem{case}[theorem]{Case}
\newtheorem{claim}[theorem]{Claim}
\newtheorem{conclusion}[theorem]{Conclusion}
\newtheorem{condition}[theorem]{Condition}
\newtheorem{conjecture}[theorem]{Conjecture}
\newtheorem{corollary}[theorem]{Corollary}
\newtheorem{criterion}[theorem]{Criterion}
\newtheorem{definition}[theorem]{Definition}
\newtheorem{example}[theorem]{Example}
\newtheorem{exercise}[theorem]{Exercise}
\newtheorem{lemma}[theorem]{Lemma}
\newtheorem{notation}[theorem]{Notation}
\newtheorem{problem}[theorem]{Problem}
\newtheorem{proposition}[theorem]{Proposition}
\newtheorem{remark}[theorem]{Remark}
\newtheorem{solution}[theorem]{Solution}
\newtheorem{summary}[theorem]{Summary}
\newenvironment{proof}[1][Proof]{\noindent\textbf{#1.} }{\ \rule{0.5em}{0.5em}}
\input{tcilatex}
\begin{document}


\bigskip
-------------------------------------------------------------------------------------------------------------------------------------------------------------------------------------

3- Sea la ecuacion diferencial de primer orden $y\prime =\frac{1-xy}{1+y^{4}}
$ con la condicion inicial $y(0)=1.2$.

para \qquad $0\leq x\leq 6\qquad $y\qquad $y(0)=1.2$

\begin{eqnarray*}
\frac{b-a}{n} &=&h \\
\frac{b-a}{h} &=&n
\end{eqnarray*}

$y^{\prime }=\frac{1-xy}{1+y^{4}}$

$a=0$

$b=6$

$n=\frac{b-a}{h}$

$h=\frac{1}{6}$

Es la distancia entre los puntos.i es la posicion, multiplicar por el h

$x_{i}=a+ih$

$\bigskip x_{i}=0+i(\frac{1}{6})$

\bigskip $f(x,y)=\frac{x-y}{1+y^{4}}$

El primer yk, lo tenemos como dato y(1)(xk)=1(yk)

$y_{k+1}=y_{k}+hf(x_{k},y_{k})$

$y_{k+1}=1.2+\frac{1}{6}(f(0,1.2))=\allowbreak 1.\,\allowbreak 134\,9$

$y_{k+1}=1.\,\allowbreak 134\,9+\frac{1}{6}(f(0.33334,1.\,\allowbreak
134\,9))=\allowbreak 1.\,\allowbreak 084\,7$

$y_{k+1}=1.\,\allowbreak 084\,7+\frac{1}{6}(f(0.50001,1.\,\allowbreak
084\,7))=\allowbreak 1.\,\allowbreak 043\,8$

$y_{k+1}=\allowbreak 1.\,\allowbreak 043\,8+\frac{1}{6}(f(0.66668,1.\,%
\allowbreak 043\,8))=\allowbreak 1.\,\allowbreak 015\,1$

$y_{k+1}=\allowbreak \allowbreak 1.\,\allowbreak 015\,1+\frac{1}{6}%
(f(0.83335,\allowbreak 1.\,\allowbreak 015\,1))=\allowbreak 1.\,\allowbreak
000\,4$

$%
\begin{array}{cc}
x_{k} & y_{k} \\ 
0 & 1.2 \\ 
0.33334 & \allowbreak 1.\,\allowbreak 134\,9 \\ 
0.50001 & \allowbreak 1.\,\allowbreak 084\,7 \\ 
0.66668 & \allowbreak 1.\,\allowbreak 043\,8 \\ 
0.83335 & 1.\,\allowbreak 015\,1 \\ 
1 & \allowbreak \allowbreak 1.\,\allowbreak 000\,4%
\end{array}%
$

Entonces:

$S(x,y)=\{(0,1.2),(0.33334,1.1349),(0.50001,\allowbreak 1.\,\allowbreak
084\,7),(0.66668,\allowbreak 1.\,\allowbreak
043\,8),(0.83335,1.\,\allowbreak 015\,1),(1,\allowbreak \allowbreak
1.\,\allowbreak 000\,4)\}$\FRAME{dtbpFX}{5.0488in}{3.3624in}{0pt}{}{}{Plot}{%
\special{language "Scientific Word";type "MAPLEPLOT";width 5.0488in;height
3.3624in;depth 0pt;display "USEDEF";plot_snapshots TRUE;mustRecompute
FALSE;lastEngine "MuPAD";xmin "-5";xmax "5";xviewmin "-0.0001";xviewmax
"1.0001";yviewmin "1";yviewmax "1.20001996";plottype 4;axesFont "Times New
Roman,12,0000000000,useDefault,normal";numpoints 100;plotstyle
"patchnogrid";axesstyle "normal";axestips FALSE;xis \TEXUX{x};var1name
\TEXUX{$x$};function
\TEXUX{$\MATRIX{2,6}{c}\VR{,,c,,,}{,,c,,,}{,,,,,}\HR{,,,,,,}\CELL{0}%
\CELL{1.2}\CELL{0.33334}\CELL{\allowbreak 1.\,\allowbreak
134\,9}\CELL{0.50001}\CELL{\allowbreak 1.\,\allowbreak
084\,7}\CELL{0.66668}\CELL{\allowbreak 1.\,\allowbreak
043\,8}\CELL{0.83335}\CELL{1.\,\allowbreak 015\,1}\CELL{1}\CELL{\allowbreak
\allowbreak 1.\,\allowbreak 000\,4}$};linecolor "blue";linestyle 1;pointplot
TRUE;pointstyle "point";linethickness 1;lineAttributes "Solid";curveColor
"[flat::RGB:0x000000ff]";curveStyle "Point";rangeset"X";VCamFile
'RDUNN307.xvz';valid_file "T";tempfilename
'RDUNMY01.wmf';tempfile-properties "XPR";}}

\bigskip
-------------------------------------------------------------------------------------------------------------------------------------------------------------------------------------

Ejercicio 4) Usando el conjunto de puntos obtenido en el ejercicio 3.
Aproxime la solucion de la ecuacion diferencial usando la familia de
funciones $A$ del ejercicio 2. Grafique.

\bigskip - Funciones del ejercicio 2:

$A=\{1,x,x^{2}\}$

$f_{1}(x)=1$

$f_{2}(x)=x$

$f_{3}(x)=x^{2}$

- Conjunto de puntos del ejercicio 3:

$S(x,y)=\{(0,1.2),(0.33334,1.1349),(0.50001,\allowbreak 1.\,\allowbreak
084\,7),(0.66668,\allowbreak 1.\,\allowbreak
043\,8),(0.83335,1.\,\allowbreak 015\,1),(1,\allowbreak \allowbreak
1.\,\allowbreak 000\,4)\}$

\bigskip

- Forma que tiene que tener el polinomio resultante:

\[
f(x)=a_{0}+a_{1}x+a_{2}x^{2} 
\]

- Columna 1: Reemplazamos cada valor de x con A=\{1\}

- Columna 2: Reemplazamos cada valor de x con A=\{x\}

- Columna 3: Reemplazamos cada valor de x con A=\{x2\}

1. Armamos la matriz A:

$A=\left[ 
\begin{array}{ccc}
1 & 0 & 0 \\ 
1 & 0.33334 & 0,11115556 \\ 
1 & 0.50001 & 0.25 \\ 
1 & 0.66668 & 0.44 \\ 
1 & 0.83335 & 0.694 \\ 
1 & 1 & 1%
\end{array}%
\right] $

\bigskip 2. Armamos la matriz Y

Columna: Valores de Y en S:

$Y=\left[ 
\begin{array}{c}
1.2 \\ 
1.1349 \\ 
1.\,\allowbreak 084\,7 \\ 
1.\,\allowbreak 043\,8 \\ 
1.\,\allowbreak 015\,1 \\ 
1.\,\allowbreak 000\,4%
\end{array}%
\right] $

3. Calculamos la transpuesta de A:

$A$, transpose: $\left[ 
\begin{array}{cccccc}
1 & 1 & 1 & 1 & 1 & 1 \\ 
1 & 1.\,\allowbreak 2 & 1.\,\allowbreak 4 & 1.\,\allowbreak 6 & 
1.\,\allowbreak 8 & 2.0 \\ 
1 & 1.\,\allowbreak 44 & 1.\,\allowbreak 96 & 2.\,\allowbreak 56 & 
3.\,\allowbreak 24 & 4%
\end{array}%
\right] \allowbreak $

$A^{t}=\left[ 
\begin{array}{cccccc}
1 & 1 & 1 & 1 & 1 & 1 \\ 
1 & 1.\,\allowbreak 2 & 1.\,\allowbreak 4 & 1.\,\allowbreak 6 & 
1.\,\allowbreak 8 & 2.0 \\ 
1 & 1.\,\allowbreak 44 & 1.\,\allowbreak 96 & 2.\,\allowbreak 56 & 
3.\,\allowbreak 24 & 4%
\end{array}%
\right] $

!!Cuidado con el orden de multiplicar

4. Calculamos el producto de A(transpuesta) x A:

$C=\left[ 
\begin{array}{cccccc}
1 & 1 & 1 & 1 & 1 & 1 \\ 
1 & 1.\,\allowbreak 2 & 1.\,\allowbreak 4 & 1.\,\allowbreak 6 & 
1.\,\allowbreak 8 & 2.0 \\ 
1 & 1.\,\allowbreak 44 & 1.\,\allowbreak 96 & 2.\,\allowbreak 56 & 
3.\,\allowbreak 24 & 4%
\end{array}%
\right] \left[ 
\begin{array}{ccc}
1 & 1 & 1 \\ 
1 & 1.2 & 1.44 \\ 
1 & 1.4 & 1.96 \\ 
1 & 1.6 & 2.56 \\ 
1 & 1.8 & 3.24 \\ 
1 & 2.0 & 4%
\end{array}%
\right] =\allowbreak \left[ 
\begin{array}{ccc}
6.0 & 9.0 & 14.\,\allowbreak 2 \\ 
9.0 & 14.\,\allowbreak 2 & 23.\,\allowbreak 4 \\ 
14.\,\allowbreak 2 & 23.\,\allowbreak 4 & 39.\,\allowbreak 966%
\end{array}%
\right] \allowbreak $

\bigskip

5. Calculamos el producto de A(transpuesta) x Y:

$D=\left[ 
\begin{array}{cccccc}
1 & 1 & 1 & 1 & 1 & 1 \\ 
1 & 1.\,\allowbreak 2 & 1.\,\allowbreak 4 & 1.\,\allowbreak 6 & 
1.\,\allowbreak 8 & 2.0 \\ 
1 & 1.\,\allowbreak 44 & 1.\,\allowbreak 96 & 2.\,\allowbreak 56 & 
3.\,\allowbreak 24 & 4%
\end{array}%
\right] \left[ 
\begin{array}{c}
1 \\ 
1 \\ 
1.\,\allowbreak 016\,4 \\ 
1.\,\allowbreak 041\,8 \\ 
1.\,\allowbreak 071\,3 \\ 
1.\,\allowbreak 102\,2%
\end{array}%
\right] =\allowbreak \left[ 
\begin{array}{c}
6.\,\allowbreak 231\,7 \\ 
9.\,\allowbreak 422\,6 \\ 
14.\,\allowbreak 979%
\end{array}%
\right] $

\bigskip

6. Multiplicamos C$^{-1}$ x D:

\bigskip $R=\left[ 
\begin{array}{ccc}
6.0 & 9.0 & 14.\,\allowbreak 2 \\ 
9.0 & 14.\,\allowbreak 2 & 23.\,\allowbreak 4 \\ 
14.\,\allowbreak 2 & 23.\,\allowbreak 4 & 39.\,\allowbreak 966%
\end{array}%
\right] ^{-1}\left[ 
\begin{array}{c}
6.\,\allowbreak 231\,7 \\ 
9.\,\allowbreak 422\,6 \\ 
14.\,\allowbreak 979%
\end{array}%
\right] =\allowbreak \left[ 
\begin{array}{c}
1.\,\allowbreak 075\,3 \\ 
-0.170\,54 \\ 
9.\,\allowbreak 258\,4\times 10^{-2}%
\end{array}%
\right] \allowbreak 
\begin{array}{c}
a0 \\ 
a1 \\ 
a2%
\end{array}%
$

\bigskip 7. Reemplazamos R en la f(x) con los a sub n.

$A=\{1,x,x^{2}\}$

$f(x)=a_{0}+a_{1}x+a_{2}x^{2}$

\[
f(x)=1.0753-0.17054x+9.\,\allowbreak 258\,4\times 10^{-2}x^{2} 
\]

\end{document}
