
\documentclass{article}
%%%%%%%%%%%%%%%%%%%%%%%%%%%%%%%%%%%%%%%%%%%%%%%%%%%%%%%%%%%%%%%%%%%%%%%%%%%%%%%%%%%%%%%%%%%%%%%%%%%%%%%%%%%%%%%%%%%%%%%%%%%%%%%%%%%%%%%%%%%%%%%%%%%%%%%%%%%%%%%%%%%%%%%%%%%%%%%%%%%%%%%%%%%%%%%%%%%%%%%%%%%%%%%%%%%%%%%%%%%%%%%%%%%%%%%%%%%%%%%%%%%%%%%%%%%%
%TCIDATA{OutputFilter=LATEX.DLL}
%TCIDATA{Version=5.50.0.2953}
%TCIDATA{<META NAME="SaveForMode" CONTENT="1">}
%TCIDATA{BibliographyScheme=Manual}
%TCIDATA{Created=Monday, June 13, 2022 20:17:50}
%TCIDATA{LastRevised=Monday, June 20, 2022 20:23:13}
%TCIDATA{<META NAME="GraphicsSave" CONTENT="32">}
%TCIDATA{<META NAME="DocumentShell" CONTENT="Standard LaTeX\Blank - Standard LaTeX Article">}
%TCIDATA{CSTFile=40 LaTeX article.cst}
%TCIDATA{ComputeDefs=
%1$\bigskip x_{i}=1+i(0.2)$
%$f(x,y)=\frac{x-y}{1+(xy)^{2}}$
%$h=\frac{2-1}{5}=0.2$
%$Y=\left[ 
%\begin{array}{c}
%1 \\ 
%1 \\ 
%1.\,\allowbreak 016\,4 \\ 
%1.\,\allowbreak 041\,8 \\ 
%1.\,\allowbreak 071\,3 \\ 
%1.\,\allowbreak 102\,2%
%\end{array}%
%\right] $
%$A^{t}=\left[ 
%\begin{array}{cccccc}
%1 & 1 & 1 & 1 & 1 & 1 \\ 
%1 & 1.\,\allowbreak 2 & 1.\,\allowbreak 4 & 1.\,\allowbreak 6 & 
%1.\,\allowbreak 8 & 2.0 \\ 
%1 & 1.\,\allowbreak 44 & 1.\,\allowbreak 96 & 2.\,\allowbreak 56 & 
%3.\,\allowbreak 24 & 4%
%\end{array}%
%\right] $
%$C=\left[ 
%\begin{array}{cccccc}
%1 & 1 & 1 & 1 & 1 & 1 \\ 
%1 & 1.\,\allowbreak 2 & 1.\,\allowbreak 4 & 1.\,\allowbreak 6 & 
%1.\,\allowbreak 8 & 2.0 \\ 
%1 & 1.\,\allowbreak 44 & 1.\,\allowbreak 96 & 2.\,\allowbreak 56 & 
%3.\,\allowbreak 24 & 4%
%\end{array}%
%\right] \left[ 
%\begin{array}{ccc}
%1 & 1 & 1 \\ 
%1 & 1.2 & 1.44 \\ 
%1 & 1.4 & 1.96 \\ 
%1 & 1.6 & 2.56 \\ 
%1 & 1.8 & 3.24 \\ 
%1 & 2.0 & 4%
%\end{array}%
%\right] =\allowbreak \left[ 
%\begin{array}{ccc}
%6.0 & 9.0 & 14.\,\allowbreak 2 \\ 
%9.0 & 14.\,\allowbreak 2 & 23.\,\allowbreak 4 \\ 
%14.\,\allowbreak 2 & 23.\,\allowbreak 4 & 39.\,\allowbreak 966%
%\end{array}%
%\right] $
%$D=\left[ 
%\begin{array}{cccccc}
%1 & 1 & 1 & 1 & 1 & 1 \\ 
%1 & 1.\,\allowbreak 2 & 1.\,\allowbreak 4 & 1.\,\allowbreak 6 & 
%1.\,\allowbreak 8 & 2.0 \\ 
%1 & 1.\,\allowbreak 44 & 1.\,\allowbreak 96 & 2.\,\allowbreak 56 & 
%3.\,\allowbreak 24 & 4%
%\end{array}%
%\right] \left[ 
%\begin{array}{c}
%1 \\ 
%1 \\ 
%1.\,\allowbreak 016\,4 \\ 
%1.\,\allowbreak 041\,8 \\ 
%1.\,\allowbreak 071\,3 \\ 
%1.\,\allowbreak 102\,2%
%\end{array}%
%\right] =\allowbreak \left[ 
%\begin{array}{c}
%6.\,\allowbreak 231\,7 \\ 
%9.\,\allowbreak 422\,6 \\ 
%14.\,\allowbreak 979%
%\end{array}%
%\right] $
%}


\newtheorem{theorem}{Theorem}
\newtheorem{acknowledgement}[theorem]{Acknowledgement}
\newtheorem{algorithm}[theorem]{Algorithm}
\newtheorem{axiom}[theorem]{Axiom}
\newtheorem{case}[theorem]{Case}
\newtheorem{claim}[theorem]{Claim}
\newtheorem{conclusion}[theorem]{Conclusion}
\newtheorem{condition}[theorem]{Condition}
\newtheorem{conjecture}[theorem]{Conjecture}
\newtheorem{corollary}[theorem]{Corollary}
\newtheorem{criterion}[theorem]{Criterion}
\newtheorem{definition}[theorem]{Definition}
\newtheorem{example}[theorem]{Example}
\newtheorem{exercise}[theorem]{Exercise}
\newtheorem{lemma}[theorem]{Lemma}
\newtheorem{notation}[theorem]{Notation}
\newtheorem{problem}[theorem]{Problem}
\newtheorem{proposition}[theorem]{Proposition}
\newtheorem{remark}[theorem]{Remark}
\newtheorem{solution}[theorem]{Solution}
\newtheorem{summary}[theorem]{Summary}
\newenvironment{proof}[1][Proof]{\noindent\textbf{#1.} }{\ \rule{0.5em}{0.5em}}
\input{tcilatex}
\begin{document}


3- Sea la ecuacion diferencial de primer orden $y\prime =\frac{x-y}{%
1+(xy)^{2}}$ para

$1\leq x\leq 2\qquad $y $\qquad y(1)=1$

Obtenga una sucesion de puntos que aproxime la solucion en el intervalo [1,2]

usando el metodo de euler con h=0.2

$\bigskip $

$y^{\prime }=\frac{x-y}{1+(xy)^{2}}\qquad $para \qquad $1\leq x\leq 2\qquad $%
y\qquad $y(1)=1$

$n=5$

$h=\frac{2-1}{5}=0.2\qquad \qquad $

Es la distancia entre los puntos.i es la posicion, multiplicar por el h

$x_{i}=a+ih$

$\bigskip x_{i}=1+i(0.2)$

\bigskip $f(x,y)=\frac{x-y}{1+(xy)^{2}}$

El primer yk, lo tenemos como dato y(0)(x)=1(y)

$y_{k+1}=y_{k}+hf(x_{k},y_{k})$

$y_{k+1}=1+0.2(f(1,1))=\allowbreak 1.0$

$y_{k+1}=1+0.2(f(1.2,1))=\allowbreak 1.\,\allowbreak 016\,4$

$y_{k+1}=\allowbreak 1.\,\allowbreak 016\,4\allowbreak
+0.2(f(1.4,\allowbreak 1.\,\allowbreak 016\,4\allowbreak ))=\allowbreak
1.\,\allowbreak 041\,8$

$y_{k+1}=\allowbreak 1.\,\allowbreak 041\,8+0.2(f(1.6,\allowbreak
1.\,\allowbreak 041\,8))=\allowbreak 1.\,\allowbreak 071\,3$

$y_{k+1}=1.\,\allowbreak 071\,3+0.2(f(1.8,\allowbreak 1.\,\allowbreak
071\,3))=\allowbreak 1.\,\allowbreak 102\,2$

$%
\begin{array}{cc}
x_{k} & y_{k} \\ 
1 & 1 \\ 
1.2 & \allowbreak 1.0 \\ 
1.4 & \allowbreak 1.\,\allowbreak 016\,4 \\ 
1.6 & 1.\,\allowbreak 041\,8 \\ 
1.8 & 1.\,\allowbreak 071\,3 \\ 
2 & \allowbreak 1.\,\allowbreak 102\,2%
\end{array}%
$

Entonces:

$S(x,y)=\{(1,1),(1.2,\allowbreak 1.0),(1.4,\allowbreak 1.\,\allowbreak
016\,4),(1.6,1.\,\allowbreak 041\,8),(1.8,1.\,\allowbreak
071\,3),(2,\allowbreak 1.\,\allowbreak 102\,2)\}$\FRAME{dtbpFX}{5.0488in}{%
3.3624in}{0pt}{}{}{Plot}{\special{language "Scientific Word";type
"MAPLEPLOT";width 5.0488in;height 3.3624in;depth 0pt;display
"USEDEF";plot_snapshots TRUE;mustRecompute FALSE;lastEngine "MuPAD";xmin
"-5";xmax "5";xviewmin "0.99989999989998";xviewmax
"2.00010000010002";yviewmin "0.997333278788998";yviewmax
"1.10456672121";plottype 4;axesFont "Times New
Roman,12,0000000000,useDefault,normal";numpoints 100;plotstyle
"patchnogrid";axesstyle "normal";axestips FALSE;xis \TEXUX{x};var1name
\TEXUX{$x$};function
\TEXUX{$\MATRIX{2,6}{c}\VR{,,c,,,}{,,c,,,}{,,,,,}\HR{,,,,,,}\CELL{1}\CELL{1}%
\CELL{1.2}\CELL{\allowbreak 1.0}\CELL{1.4}\CELL{\allowbreak 1.\,\allowbreak
016\,4}\CELL{1.6}\CELL{1.\,\allowbreak
041\,8}\CELL{1.8}\CELL{1.\,\allowbreak 071\,3}\CELL{2}\CELL{\allowbreak
1.\,\allowbreak 102\,2}$};linecolor "blue";linestyle 1;pointplot
TRUE;pointstyle "point";linethickness 1;lineAttributes "Solid";curveColor
"[flat::RGB:0x000000ff]";curveStyle "Point";rangeset"X";function
\TEXUX{$1.0753-0.17054x+9.\,\allowbreak 258\,4\times
10^{-2}x^{2}$};linecolor "red";linestyle 1;pointstyle "point";linethickness
2;lineAttributes "Solid";var1range "1,2";num-x-gridlines 100;curveColor
"[flat::RGB:0x00ff0000]";curveStyle "Line";rangeset"X";VCamFile
'RDSU9E0E.xvz';valid_file "T";tempfilename
'RDSU8K02.wmf';tempfile-properties "XPR";}}

\bigskip 

Ejercicio 4) Usando el conjunto de puntos obtenido en el ejercicio 3.
Aproxime la solucion de la ecuacion diferencial usando la familia de

funciones A del ejercicio 2. Grafique.

\bigskip Funciones del ejercicio 2:

$A=\{1,x,x^{2}\}$

$f_{1}(x)=1$

$f_{2}(x)=x$

$f_{3}(x)=x^{2}$

Conjunto de puntos del ejercicio 3:

$S(x,y)=\{(1,1),(1.2,\allowbreak 1.0),(1.4,\allowbreak 1.\,\allowbreak
016\,4),(1.6,1.\,\allowbreak 041\,8),(1.8,1.\,\allowbreak
071\,3),(2,\allowbreak 1.\,\allowbreak 102\,2)\}$

\bigskip 

Forma que tiene que tener el polinomio resultante:

\[
f(x)=a_{0}+a_{1}x+a_{2}x^{2}
\]

1- Columna 1: Reemplazamos cada valor de x con A=\{1\}

2- Columna 2: Reemplazamos cada valor de x con A=\{x\}

3- Columna 3: Reemplazamos cada valor de x con A=\{x2\}

$A=\left[ 
\begin{array}{ccc}
1 & 1 & 1 \\ 
1 & 1.2 & 1.44 \\ 
1 & 1.4 & 1.96 \\ 
1 & 1.6 & 2.56 \\ 
1 & 1.8 & 3.24 \\ 
1 & 2.0 & 4%
\end{array}%
\right] $

\bigskip Columna: Valores de Y en S:

$Y=\left[ 
\begin{array}{c}
1 \\ 
1 \\ 
1.\,\allowbreak 016\,4 \\ 
1.\,\allowbreak 041\,8 \\ 
1.\,\allowbreak 071\,3 \\ 
1.\,\allowbreak 102\,2%
\end{array}%
\right] $

\bigskip 

Calculamos la transpuesta de A:

$A$, transpose: $\left[ 
\begin{array}{cccccc}
1 & 1 & 1 & 1 & 1 & 1 \\ 
1 & 1.\,\allowbreak 2 & 1.\,\allowbreak 4 & 1.\,\allowbreak 6 & 
1.\,\allowbreak 8 & 2.0 \\ 
1 & 1.\,\allowbreak 44 & 1.\,\allowbreak 96 & 2.\,\allowbreak 56 & 
3.\,\allowbreak 24 & 4%
\end{array}%
\right] \allowbreak $

$A^{t}=\left[ 
\begin{array}{cccccc}
1 & 1 & 1 & 1 & 1 & 1 \\ 
1 & 1.\,\allowbreak 2 & 1.\,\allowbreak 4 & 1.\,\allowbreak 6 & 
1.\,\allowbreak 8 & 2.0 \\ 
1 & 1.\,\allowbreak 44 & 1.\,\allowbreak 96 & 2.\,\allowbreak 56 & 
3.\,\allowbreak 24 & 4%
\end{array}%
\right] $

!!Cuidado con el orden de multiplicar

Calculamos el producto de A(transpuesta) x A:

$C=\left[ 
\begin{array}{cccccc}
1 & 1 & 1 & 1 & 1 & 1 \\ 
1 & 1.\,\allowbreak 2 & 1.\,\allowbreak 4 & 1.\,\allowbreak 6 & 
1.\,\allowbreak 8 & 2.0 \\ 
1 & 1.\,\allowbreak 44 & 1.\,\allowbreak 96 & 2.\,\allowbreak 56 & 
3.\,\allowbreak 24 & 4%
\end{array}%
\right] \left[ 
\begin{array}{ccc}
1 & 1 & 1 \\ 
1 & 1.2 & 1.44 \\ 
1 & 1.4 & 1.96 \\ 
1 & 1.6 & 2.56 \\ 
1 & 1.8 & 3.24 \\ 
1 & 2.0 & 4%
\end{array}%
\right] =\allowbreak \left[ 
\begin{array}{ccc}
6.0 & 9.0 & 14.\,\allowbreak 2 \\ 
9.0 & 14.\,\allowbreak 2 & 23.\,\allowbreak 4 \\ 
14.\,\allowbreak 2 & 23.\,\allowbreak 4 & 39.\,\allowbreak 966%
\end{array}%
\right] \allowbreak $

\bigskip 

Calculamos el producto de A(transpuesta) x Y:

$D=\left[ 
\begin{array}{cccccc}
1 & 1 & 1 & 1 & 1 & 1 \\ 
1 & 1.\,\allowbreak 2 & 1.\,\allowbreak 4 & 1.\,\allowbreak 6 & 
1.\,\allowbreak 8 & 2.0 \\ 
1 & 1.\,\allowbreak 44 & 1.\,\allowbreak 96 & 2.\,\allowbreak 56 & 
3.\,\allowbreak 24 & 4%
\end{array}%
\right] \left[ 
\begin{array}{c}
1 \\ 
1 \\ 
1.\,\allowbreak 016\,4 \\ 
1.\,\allowbreak 041\,8 \\ 
1.\,\allowbreak 071\,3 \\ 
1.\,\allowbreak 102\,2%
\end{array}%
\right] =\allowbreak \left[ 
\begin{array}{c}
6.\,\allowbreak 231\,7 \\ 
9.\,\allowbreak 422\,6 \\ 
14.\,\allowbreak 979%
\end{array}%
\right] $

\bigskip 

Multiplicamos C$^{-1}$ x D:

\bigskip $R=\left[ 
\begin{array}{ccc}
6.0 & 9.0 & 14.\,\allowbreak 2 \\ 
9.0 & 14.\,\allowbreak 2 & 23.\,\allowbreak 4 \\ 
14.\,\allowbreak 2 & 23.\,\allowbreak 4 & 39.\,\allowbreak 966%
\end{array}%
\right] ^{-1}\left[ 
\begin{array}{c}
6.\,\allowbreak 231\,7 \\ 
9.\,\allowbreak 422\,6 \\ 
14.\,\allowbreak 979%
\end{array}%
\right] =\allowbreak \left[ 
\begin{array}{c}
1.\,\allowbreak 075\,3 \\ 
-0.170\,54 \\ 
9.\,\allowbreak 258\,4\times 10^{-2}%
\end{array}%
\right] \allowbreak 
\begin{array}{c}
a0 \\ 
a1 \\ 
a2%
\end{array}%
$

\bigskip Reemplazamos R en la f(x) con los a sub n.

$A=\{1,x,x^{2}\}$

$f(x)=a_{0}+a_{1}x+a_{2}x^{2}$

\[
f(x)=1.0753-0.17054x+9.\,\allowbreak 258\,4\times 10^{-2}x^{2}
\]

\end{document}
