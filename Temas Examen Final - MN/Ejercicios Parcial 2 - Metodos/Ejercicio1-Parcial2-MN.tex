
\documentclass{article}
%%%%%%%%%%%%%%%%%%%%%%%%%%%%%%%%%%%%%%%%%%%%%%%%%%%%%%%%%%%%%%%%%%%%%%%%%%%%%%%%%%%%%%%%%%%%%%%%%%%%%%%%%%%%%%%%%%%%%%%%%%%%%%%%%%%%%%%%%%%%%%%%%%%%%%%%%%%%%%%%%%%%%%%%%%%%%%%%%%%%%%%%%%%%%%%%%%%%%%%%%%%%%%%%%%%%%%%%%%%%%%%%%%%%%%%%%%%%%%%%%%%%%%%%%%%%
%TCIDATA{OutputFilter=LATEX.DLL}
%TCIDATA{Version=5.50.0.2953}
%TCIDATA{<META NAME="SaveForMode" CONTENT="1">}
%TCIDATA{BibliographyScheme=Manual}
%TCIDATA{Created=Monday, June 20, 2022 20:27:15}
%TCIDATA{LastRevised=Tuesday, June 21, 2022 08:47:15}
%TCIDATA{<META NAME="GraphicsSave" CONTENT="32">}
%TCIDATA{<META NAME="DocumentShell" CONTENT="Standard LaTeX\Blank - Standard LaTeX Article">}
%TCIDATA{CSTFile=40 LaTeX article.cst}
%TCIDATA{ComputeDefs=
%$n=14$
%$a=0$
%$b=1$
%$f(x)=(1+x^{2})^{x}$
%$h=\frac{b-a}{n}=\allowbreak 7.\,\allowbreak 142\,9\times 10^{-2}$
%}


\newtheorem{theorem}{Theorem}
\newtheorem{acknowledgement}[theorem]{Acknowledgement}
\newtheorem{algorithm}[theorem]{Algorithm}
\newtheorem{axiom}[theorem]{Axiom}
\newtheorem{case}[theorem]{Case}
\newtheorem{claim}[theorem]{Claim}
\newtheorem{conclusion}[theorem]{Conclusion}
\newtheorem{condition}[theorem]{Condition}
\newtheorem{conjecture}[theorem]{Conjecture}
\newtheorem{corollary}[theorem]{Corollary}
\newtheorem{criterion}[theorem]{Criterion}
\newtheorem{definition}[theorem]{Definition}
\newtheorem{example}[theorem]{Example}
\newtheorem{exercise}[theorem]{Exercise}
\newtheorem{lemma}[theorem]{Lemma}
\newtheorem{notation}[theorem]{Notation}
\newtheorem{problem}[theorem]{Problem}
\newtheorem{proposition}[theorem]{Proposition}
\newtheorem{remark}[theorem]{Remark}
\newtheorem{solution}[theorem]{Solution}
\newtheorem{summary}[theorem]{Summary}
\newenvironment{proof}[1][Proof]{\noindent\textbf{#1.} }{\ \rule{0.5em}{0.5em}}
\input{tcilatex}
\begin{document}


\bigskip
------------------------------------------------------------------------------------------------------------------------------------------------------------------------------

Simpson Compuesto.

1. Use el metodo de Sinson compuesto para aproximar el valor de $%
I=\int_{0}^{1}(1+x^{2})^{x}dx$. Tome n=14 y calcule una cota para el error.

$I=\int_{0}^{1}(1+x^{2})^{x}dx=\allowbreak 1.\,\allowbreak 240\,2$

$f(x)=(1+x^{2})^{x}$

$n=14$

$a=0\qquad $x0 = a

$b=1\qquad $xn =\ b

$h=\frac{b-a}{n}=\allowbreak 7.\,\allowbreak 142\,9\times 10^{-2}$

\[
I(f)=\frac{h}{3}\left[ f(x_{0})+4\sum\limits_{j=0}^{\frac{n}{2}%
-1}f(x_{2j+1})+2\sum\limits_{j=1}^{\frac{n}{2}-1}f(x_{2j})+f(x_{n})\right] 
\]

$I(f)=\frac{h}{3}\left[ f(0)+4\sum\limits_{j=0}^{6}(f(x_{2j+1}))+2\sum%
\limits_{j=1}^{6}(f(x_{2j}))+f(1)\right] $

\bigskip $x_{0}=0$

$x_{1}=0+h=\allowbreak 7.\,\allowbreak 142\,9\times 10^{-2}$

$x_{2}=0+2h=\allowbreak \frac{1}{7}=0.142\,86$

$x_{3}=0+3h=\allowbreak \frac{3}{14}=0.214\,29$

$x_{4}=0+4h=\allowbreak \frac{2}{7}=0.285\,72$

$x_{5}=0+5h=\allowbreak \frac{5}{14}=0.357\,15$

$x_{6}=0+6h=\allowbreak \frac{3}{7}=0.428\,57$

$x_{7}=0+7h=\allowbreak \frac{1}{2}=0.5$

$x_{8}=0+8h=\allowbreak \frac{4}{7}=0.571\,43$

$x_{9}=0+9h=\allowbreak \frac{9}{14}=0.642\,86$

$x_{10}=0+10h=\allowbreak \frac{5}{7}=0.714\,29$

$x_{11}=0+11h=\allowbreak \frac{11}{14}=0.785\,72$

$x_{12}=0+12h=\allowbreak \frac{6}{7}=0.857\,15$

$x_{13}=0+13h=\allowbreak \frac{13}{14}=0.928\,58$

$x_{14}=0+14h=\allowbreak 1=1.0$

\QTP{Body Math}
\[
I(f)=\frac{h}{3}\left[
f(x_{0})+4f(x_{1})+2f(x_{2})+4f(x_{3})+2f(x_{4})+4f(x_{5})+2f(x_{6})+4f(x_{7})+2f(x_{8})+4f(x_{9})+2f(x_{10})+4f(x_{11})+2f(x_{12})+4f(x_{13})+f(x_{14})%
\right] 
\]

\QTP{Body Math}
$\qquad I(f)=\frac{h}{3}%
(f(0)+4f(1h)+2f(2h)+4f(3h)+2f(4h)+4f(5h)+2f(6h)+4f(7h)+2f(8h)+4f(9h)+2f(10h)+4f(11h)+2f(12h)+4f(13h)+f(1))\allowbreak 
$ : $1.\,\allowbreak 240\,2=1.\,\allowbreak 240\,2\allowbreak $

\QTP{Body Math}
$\qquad $

\end{document}
