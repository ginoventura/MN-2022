
\documentclass{article}
%%%%%%%%%%%%%%%%%%%%%%%%%%%%%%%%%%%%%%%%%%%%%%%%%%%%%%%%%%%%%%%%%%%%%%%%%%%%%%%%%%%%%%%%%%%%%%%%%%%%%%%%%%%%%%%%%%%%%%%%%%%%%%%%%%%%%%%%%%%%%%%%%%%%%%%%%%%%%%%%%%%%%%%%%%%%%%%%%%%%%%%%%%%%%%%%%%%%%%%%%%%%%%%%%%%%%%%%%%%%%%%%%%%%%%%%%%%%%%%%%%%%%%%%%%%%
%TCIDATA{OutputFilter=LATEX.DLL}
%TCIDATA{Version=5.50.0.2953}
%TCIDATA{<META NAME="SaveForMode" CONTENT="1">}
%TCIDATA{BibliographyScheme=Manual}
%TCIDATA{Created=Monday, June 13, 2022 14:16:00}
%TCIDATA{LastRevised=Tuesday, June 21, 2022 08:46:31}
%TCIDATA{<META NAME="GraphicsSave" CONTENT="32">}
%TCIDATA{<META NAME="DocumentShell" CONTENT="Standard LaTeX\Blank - Standard LaTeX Article">}
%TCIDATA{CSTFile=40 LaTeX article.cst}
%TCIDATA{ComputeDefs=
%$f(x)=(1+x^{2})^{\frac{1}{1+x^{2}}}$
%$a=0$
%$b=1$
%}


\newtheorem{theorem}{Theorem}
\newtheorem{acknowledgement}[theorem]{Acknowledgement}
\newtheorem{algorithm}[theorem]{Algorithm}
\newtheorem{axiom}[theorem]{Axiom}
\newtheorem{case}[theorem]{Case}
\newtheorem{claim}[theorem]{Claim}
\newtheorem{conclusion}[theorem]{Conclusion}
\newtheorem{condition}[theorem]{Condition}
\newtheorem{conjecture}[theorem]{Conjecture}
\newtheorem{corollary}[theorem]{Corollary}
\newtheorem{criterion}[theorem]{Criterion}
\newtheorem{definition}[theorem]{Definition}
\newtheorem{example}[theorem]{Example}
\newtheorem{exercise}[theorem]{Exercise}
\newtheorem{lemma}[theorem]{Lemma}
\newtheorem{notation}[theorem]{Notation}
\newtheorem{problem}[theorem]{Problem}
\newtheorem{proposition}[theorem]{Proposition}
\newtheorem{remark}[theorem]{Remark}
\newtheorem{solution}[theorem]{Solution}
\newtheorem{summary}[theorem]{Summary}
\newenvironment{proof}[1][Proof]{\noindent\textbf{#1.} }{\ \rule{0.5em}{0.5em}}
\input{tcilatex}
\begin{document}


\bigskip Chebichev

2) Aproxime la funcion del ejercicio 1 usando los poliniomios de CHEBICHEV
hasta grado 2.

\bigskip $f(x)=(1+x^{2})^{\frac{1}{1+x^{2}}}\qquad \qquad $intervalo: $[0,1]$

$a=0$

$b=1$

$T_{0}=1\qquad \qquad \qquad \qquad \qquad $Inventar si no las da, desde 0

$T_{1}=x\qquad \qquad \qquad \qquad \qquad $hasta el grado de dato.

$T_{2}=2x^{2}-1$

\qquad \qquad \qquad \qquad \qquad \qquad \qquad \qquad \qquad \qquad \qquad
\qquad \qquad \qquad \qquad \qquad \qquad \qquad !!!!Formula:

\[
g(x)=\frac{1}{b-a}(2x-a-b)=\frac{1}{1-0}(2x-0-1)=\allowbreak 2x-1 
\]

1- Reemplazar g(x) en los T.

$T_{g(x)0}=1$

$T_{g(x)1}=2x-1$

$T_{g(x)2}=2(2x-1)^{2}-1$

2- Calcular los C. La integral va siempre de 0 a $\pi .$

$c_{0}=\frac{1}{\pi }\int_{0}^{\pi }f(\frac{1}{2}(b-a)\cos (\theta )+\frac{%
a+b}{2})\cos (0\theta )d\theta =\allowbreak 1.\,\allowbreak 199\,0$

$c_{1}=\frac{2}{\pi }\int_{0}^{\pi }f(\frac{1}{2}(b-a)\cos (\theta )+\frac{%
a+b}{2})\cos (1\theta )d\theta =\allowbreak 0.227\,85$

$c_{2}=\frac{2}{\pi }\int_{0}^{\pi }f(\frac{1}{2}(b-a)\cos (\theta )+\frac{%
a+b}{2})\cos (2\theta )d\theta =\allowbreak 6.\,\allowbreak 021\,4\times
10^{-3}$

$\bigskip $3- Hacer el polinomio h. !!Usar la formula de h(x)

\[
h(x)=c_{0}(T_{g(x)0})+c_{1}(T_{g(x)1})+c_{2}(T_{g(x)2}) 
\]

\[
h(x)=1.\,\allowbreak 199\,0(1)+\allowbreak 0.227\,85(2x-1)+6.\,\allowbreak
021\,4\times 10^{-3}(2(2x-1)^{2}-1) 
\]

$f(x)$\FRAME{dtbpFX}{4.4996in}{3in}{0pt}{}{}{Plot}{\special{language
"Scientific Word";type "MAPLEPLOT";width 4.4996in;height 3in;depth
0pt;display "USEDEF";plot_snapshots TRUE;mustRecompute FALSE;lastEngine
"MuPAD";xmin "0";xmax "1";xviewmin "-0.00010000010002";xviewmax
"1.00010000010002";yviewmin "0.97712582995399";yviewmax
"1.43291697005001";plottype 4;axesFont "Times New
Roman,12,0000000000,useDefault,normal";numpoints 100;plotstyle
"patch";axesstyle "normal";axestips FALSE;xis \TEXUX{x};var1name
\TEXUX{$x$};function \TEXUX{$f(x)$};linecolor "black";linestyle 1;pointstyle
"point";linethickness 1;lineAttributes "Solid";var1range
"0,1";num-x-gridlines 100;curveColor "[flat::RGB:0000000000]";curveStyle
"Line";rangeset"X";function \TEXUX{$1.\,\allowbreak 199\,0(1)+\allowbreak
0.227\,85(2x-1)+6.\,\allowbreak 021\,4\times
10^{-3}(2(2x-1)^{2}-1)$};linecolor "yellow";linestyle 1;pointstyle
"point";linethickness 1;lineAttributes "Solid";var1range
"0,1";num-x-gridlines 100;curveColor "[flat::RGB:0x00ffff00]";curveStyle
"Line";rangeset"X";VCamFile 'RDFGMT09.xvz';valid_file "T";tempfilename
'RDFGLK02.wmf';tempfile-properties "XPR";}}

\end{document}
