
\documentclass{article}
%%%%%%%%%%%%%%%%%%%%%%%%%%%%%%%%%%%%%%%%%%%%%%%%%%%%%%%%%%%%%%%%%%%%%%%%%%%%%%%%%%%%%%%%%%%%%%%%%%%%%%%%%%%%%%%%%%%%%%%%%%%%%%%%%%%%%%%%%%%%%%%%%%%%%%%%%%%%%%%%%%%%%%%%%%%%%%%%%%%%%%%%%%%%%%%%%%%%%%%%%%%%%%%%%%%%%%%%%%%%%%%%%%%%%%%%%%%%%%%%%%%%%%%%%%%%
%TCIDATA{OutputFilter=LATEX.DLL}
%TCIDATA{Version=5.50.0.2953}
%TCIDATA{<META NAME="SaveForMode" CONTENT="1">}
%TCIDATA{BibliographyScheme=Manual}
%TCIDATA{Created=Saturday, June 11, 2022 14:53:29}
%TCIDATA{LastRevised=Tuesday, June 14, 2022 20:53:17}
%TCIDATA{<META NAME="GraphicsSave" CONTENT="32">}
%TCIDATA{<META NAME="DocumentShell" CONTENT="Standard LaTeX\Blank - Standard LaTeX Article">}
%TCIDATA{CSTFile=40 LaTeX article.cst}
%TCIDATA{ComputeDefs=
%$I(x)=(1+x^{2})^{x}$
%$a=1$
%$b=0$
%}


\newtheorem{theorem}{Theorem}
\newtheorem{acknowledgement}[theorem]{Acknowledgement}
\newtheorem{algorithm}[theorem]{Algorithm}
\newtheorem{axiom}[theorem]{Axiom}
\newtheorem{case}[theorem]{Case}
\newtheorem{claim}[theorem]{Claim}
\newtheorem{conclusion}[theorem]{Conclusion}
\newtheorem{condition}[theorem]{Condition}
\newtheorem{conjecture}[theorem]{Conjecture}
\newtheorem{corollary}[theorem]{Corollary}
\newtheorem{criterion}[theorem]{Criterion}
\newtheorem{definition}[theorem]{Definition}
\newtheorem{example}[theorem]{Example}
\newtheorem{exercise}[theorem]{Exercise}
\newtheorem{lemma}[theorem]{Lemma}
\newtheorem{notation}[theorem]{Notation}
\newtheorem{problem}[theorem]{Problem}
\newtheorem{proposition}[theorem]{Proposition}
\newtheorem{remark}[theorem]{Remark}
\newtheorem{solution}[theorem]{Solution}
\newtheorem{summary}[theorem]{Summary}
\newenvironment{proof}[1][Proof]{\noindent\textbf{#1.} }{\ \rule{0.5em}{0.5em}}
\input{tcilatex}
\begin{document}


\ \ \ \ \ \ \ \ \ \ \ \ \ \ \ \ \ \ \ \ \ \ \ \ \ \ \ \ \ \ \ \ \ \ \ \ \ \
\ \ \ \ \ \ \ \ \ \ \ \ \ \ \ \ \ \ \ \ \ \ \ \ \ \ \ \ \ \ \ \ \ \ \ \ \ \
\ \ \ \ \ METODO DE NEWTON\ \ \ \ \ \ \ \ \ \ \ \ \ \ \ \ \ \ \ \ \ \ \ \ \
\ \ \ \ \ \ \ \ \ \ \ \ \ \ \ \ \ \ \ \ \ 
\begin{eqnarray*}
\frac{b-a}{n} &=&h \\
\frac{b-a}{h} &=&n
\end{eqnarray*}

\begin{eqnarray*}
n(sub\text{ }intervalos) &=&1\rightarrow \frac{1}{2}(x_{1}-x_{0})\left[
f(x_{0})+f(x_{1})\right] \\
n &=&2\rightarrow \frac{h}{3}\left[ f(x_{0})+4f(x_{1})+f(x_{2})\right] \\
n &=&3\rightarrow \frac{3}{8}h\left[ f(x_{0})+3f(x_{1})+3f(x_{2})+f(x_{3})%
\right] \\
n &=&1\text{ cotas }\rightarrow \frac{(b-a)^{3}}{12}f^{\prime \prime }(\xi )
\\
n &=&2\text{ cotas}\rightarrow \frac{h^{5}}{90}f^{(4)}(\xi )
\end{eqnarray*}

\ \ \ \ \ \ \ \ \ \ \ \ \ \ \ \ \ \ \ \ \ \ \ \ \ \ \ \ \ \ \ \ \ \ \ \ \ \
\ \ \ \ \ \ \ \ \ \ \ \ \ \ \ \ \ \ \ \ \ \ \ \ \ \ \ \ \ \ 

\ \ \ \ \ \ \ \ \ \ \ \ \ \ \ \ \ \ \ \ \ \ \ \ \ \ \ \ \ \ \ \ \ \ \ \ \ \
\ \ \ \ \ \ \ \ \ \ \ \ \ \ \ \ \ \ \ \ \ \ \ \ \ \ \ \ \ \ \ \ \ \ \ \ \ \
\ METODO DE SIMPSON

\ \ \ \ \ \ \ \ \ \ \ \ \ \ \ 

\bigskip 1- Calcular la integral de la funcion en el intervalo para $x\in %
\left[ 0,1\right] $

$I(x)=(1+x^{2})^{x}$

$I=\int_{0}^{1}(1+x^{2})^{x}dx=\allowbreak 1.\,\allowbreak 240\,2$

$a=1$

$b=0$

2- Calcular el h

$h=\frac{0-1}{14}=\allowbreak -7.\,\allowbreak 142\,9\times 10^{-2}$

3- Reemplazar en la formula de acuerdo al n(dato) y con el h

\ \ \ \ $\frac{h}{3}\left[ f(x_{0})+4f(x_{1})+f(x_{2})\right] \qquad \qquad
\qquad \qquad \qquad \qquad \qquad $x$_{0}$ es a, x$_{2}$ es b y x$_{1}$
tiene que estar entre medio

\ \ \ \ $I_{2}(f)=\frac{2}{3}\left[ \cos (1)+4\cos (3)+\cos (5)\right]
=-2.0907$

4- Cuando el intervalo es muy grande, la aproximacion es mala, si achicamos
el intervalo $\left[ a,b\right] $, la aproximacion sera mejor

Intervalo mas chico: $\left[ 1,1.4\right] $ \ \ \ \ \ \ \ \ \ \ \ \ \ \ \ \
\ \ \ \ \ \ \ \ \ \ \ \ \ \ \ \ \ \ \ \ \ \ \ \ \ \ \ \ \ \ \ \ \ \ \ \ 

5- Integramos funcion original con el intervalo mas chico

$\int_{1}^{1.4}\cos (x)dx=\allowbreak 0.143\,98$

6- Calculamos el h nuevamente, con el intervalo mas chico

$h=\frac{(1.4-1)}{2}=\allowbreak 0.2$

7- Reemplazar en la formula de acuerdo al n(dato) y con el h nuevo

$\frac{0.2}{3}\left[ \cos (1)+4\cos (1.2)+\cos (1.4)\right] =0.14398$

Para calcular las cotas tenemos que derivar a f respecto de x,

si la n es igual a 1:\ derivamos dos veces.

si la n es igual a 2:\ derivamos cuatro veces.

\end{document}
