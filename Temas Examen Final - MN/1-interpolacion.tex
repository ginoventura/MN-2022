
\documentclass{article}
%%%%%%%%%%%%%%%%%%%%%%%%%%%%%%%%%%%%%%%%%%%%%%%%%%%%%%%%%%%%%%%%%%%%%%%%%%%%%%%%%%%%%%%%%%%%%%%%%%%%%%%%%%%%%%%%%%%%%%%%%%%%%%%%%%%%%%%%%%%%%%%%%%%%%%%%%%%%%%%%%%%%%%%%%%%%%%%%%%%%%%%%%%%%%%%%%%%%%%%%%%%%%%%%%%%%%%%%%%%%%%%%%%%%%%%%%%%%%%%%%%%%%%%%%%%%
%TCIDATA{OutputFilter=LATEX.DLL}
%TCIDATA{Version=5.50.0.2953}
%TCIDATA{<META NAME="SaveForMode" CONTENT="1">}
%TCIDATA{BibliographyScheme=Manual}
%TCIDATA{Created=Tuesday, July 12, 2022 15:52:16}
%TCIDATA{LastRevised=Tuesday, July 26, 2022 09:39:49}
%TCIDATA{<META NAME="GraphicsSave" CONTENT="32">}
%TCIDATA{<META NAME="DocumentShell" CONTENT="Standard LaTeX\Blank - Standard LaTeX Article">}
%TCIDATA{CSTFile=40 LaTeX article.cst}
%TCIDATA{ComputeDefs=
%$L_{0}(x)=\frac{(x-5)(x-6)}{(3-5)(3-6)}=\allowbreak 0.166\,67\left(
%x-5.0\right) \allowbreak \left( x-6.0\right) $
%$L_{1}(x)=\frac{(x-3)(x-6)}{(5-3)(5-6)}=\allowbreak -0.5\left( x-6.0\right)
%\left( x-3.0\right) $
%$\bigskip L_{2}(x)=\frac{(x-3)(x-5)}{(6-3)(6-5)}=\allowbreak 0.333\,33\left(
%x-5.0\right) \allowbreak \left( x-3.0\right) $
%$L(x)=4L_{0}(x)+9L_{1}(x)+12L_{2}(x)=\allowbreak 0.666\,67\left(
%x-5.0\right) \allowbreak \left( x-6.0\right) +4.0\left( x-5.0\right) \left(
%x-3.0\right) -\allowbreak 4.\,\allowbreak 5\left( x-6.0\right) \left(
%x-3.0\right) $
%}


\newtheorem{theorem}{Theorem}
\newtheorem{acknowledgement}[theorem]{Acknowledgement}
\newtheorem{algorithm}[theorem]{Algorithm}
\newtheorem{axiom}[theorem]{Axiom}
\newtheorem{case}[theorem]{Case}
\newtheorem{claim}[theorem]{Claim}
\newtheorem{conclusion}[theorem]{Conclusion}
\newtheorem{condition}[theorem]{Condition}
\newtheorem{conjecture}[theorem]{Conjecture}
\newtheorem{corollary}[theorem]{Corollary}
\newtheorem{criterion}[theorem]{Criterion}
\newtheorem{definition}[theorem]{Definition}
\newtheorem{example}[theorem]{Example}
\newtheorem{exercise}[theorem]{Exercise}
\newtheorem{lemma}[theorem]{Lemma}
\newtheorem{notation}[theorem]{Notation}
\newtheorem{problem}[theorem]{Problem}
\newtheorem{proposition}[theorem]{Proposition}
\newtheorem{remark}[theorem]{Remark}
\newtheorem{solution}[theorem]{Solution}
\newtheorem{summary}[theorem]{Summary}
\newenvironment{proof}[1][Proof]{\noindent\textbf{#1.} }{\ \rule{0.5em}{0.5em}}
\input{tcilatex}
\begin{document}


\ \ \ \ \ \ \ \ \ \ \ \ \ \ \ \ \ \ \ \ \ \ \ \ \ \ \ \ \ \ \ \ \ \ \ \ \ \
\ \ \ \ \ \ \ \ \ \ \ \ \ \ \ \ \ \ \ \ \ \ 
\[
\ \mathbf{\ Interpolaci\acute{o}n\ de\ Lagrange}
\]

\textit{Sea }$S=\left\{ x_{k,}y_{k}\right\} _{k=0}^{n}\subset $ $R^{2}$.%
\textit{\ Se trata de hallar un polinomio de grado n tal que }$P(x_{k})=y_{k}
$ \textit{para }$k=0,1,2,3,\cdot \cdot \cdot \cdot ,n$.

\[
\mathbf{Teorema} 
\]

\textit{Sea }$S=\left\{ x_{k,}y_{k}\right\} _{k=0}^{n}\subset $ $R^{2}$, 
\emph{entonces existe un \'{u}nico polinomio}\textit{\ }$P$ \textit{de grado
n tal que }$P(x_{k})=y_{k}$ \textit{para }$k=0,1,2,3,\cdot \cdot \cdot \cdot
,n$.

\[
\mathbf{Demostraci\acute{o}n} 
\]

\textit{Sea P y Q polinomios de grado n tal que }$P(x_{k})=y_{k}$ \textit{y }%
$Q(x_{k})=y_{k}$ \textit{para }$k=0,1,2,3,\cdot \cdot \cdot ,n.$

\textit{Sea el polinomio }$H(x)=P(x)-Q(x)$, \textit{entonces el grado de }$%
H\leq n$ \textit{y }$H(x_{k})=0\mathit{\ }$\textit{para }$k=0,1,2,3,\cdot
\cdot \cdot \cdot ,n$, \textit{esto es, }$H$\textit{\ tiene }$n+1$\textit{\
ra\'{\i}ces pero el grado de }$H\leq n,$\textit{\ por lo tanto} $H$ \textit{%
es el polinomio nulo, o sea}, $\ H(x)=0$ \textit{para todo} $x\in R$.

\textit{Sea }$S=\left\{ x_{k,}y_{k}\right\} _{k=0}^{n}$ \textit{tal que} $%
x_{0}<x_{1}<x_{2}<\cdot \cdot \cdot <x_{n}.$\textit{Para }$k=0,1,2,\cdot
\cdot \cdot ,n.$

\textit{Sea} $L_{k}$ \textit{el polinomio de grado n dado por:}

\[
L_{k}(x)=\frac{(x-x_{0})(x-x_{1})\cdot \cdot \cdot
(x-x_{k-1})(x-x_{k+1})\cdot \cdot \cdot (x-x_{n})}{%
(x_{k}-x_{0})(x_{k}-x_{1})\cdot \cdot \cdot
(x_{k}-x_{k-1})(x_{k}-x_{k+1})\cdot \cdot \cdot (x_{k}-x_{n})} 
\]

\textit{Los polinomios} $L_{k}$ \textit{se denominan Polinomios de Lagrange:}

\[
L_{k}(x_{j})=\{_{1\text{ si }j=k}^{0\text{ si }j\neq k} 
\]

\textit{Sea ahora el polinomio }$L(1)=\dsum%
\limits_{k=0}^{n}y_{k}L_{k}(x_{j})=y_{j}L_{j}(x_{j})=y_{j}$.

\textit{Se denomina a} $L$ \emph{el polinomio interpolante de Lagrange.}

\textit{Ejemplo:}

$\bigskip S=\left\{ \left( 3,4\right) \left( 5,9\right) \left( 6,12\right)
\right\} $

\bigskip $L_{0}(x)=\frac{(x-5)(x-6)}{(3-5)(3-6)}=\allowbreak 0.166\,67\left(
x-5.0\right) \allowbreak \left( x-6.0\right) $

$L_{1}(x)=\frac{(x-3)(x-6)}{(5-3)(5-6)}=\allowbreak -0.5\left( x-6.0\right)
\left( x-3.0\right) $

$\bigskip L_{2}(x)=\frac{(x-3)(x-5)}{(6-3)(6-5)}=\allowbreak 0.333\,33\left(
x-5.0\right) \allowbreak \left( x-3.0\right) $

$L(x)=4L_{0}(x)+9L_{1}(x)+12L_{2}(x)=\allowbreak 0.666\,67\left(
x-5.0\right) \allowbreak \left( x-6.0\right) +4.0\left( x-5.0\right) \left(
x-3.0\right) -\allowbreak 4.\,\allowbreak 5\left( x-6.0\right) \left(
x-3.0\right) $

$L(3)=\allowbreak 4.0$

\[
\mathbf{Teorema} 
\]

\textit{Sea }$F:R\rightarrow R$ \textit{de clase }$C^{n+1}$,$\left\{
x_{k}\right\} _{k=0}^{n}$ \textit{con }$x_{0}<x_{1}<x_{2}<\cdot \cdot \cdot
<x_{n}$.\textit{Si }$L$ \textit{es el polinomio interpolante de Lagrange de
grado }$\mathit{n}$\textit{\ para el conjunto }$S=\left\{ \left(
x_{k},y_{k}\right) \right\} _{k=0}^{n}$ \textit{entonce:}

\[
f(x)-L(x)=\frac{f(\xi )}{(n+1)!}\dprod\limits_{k=0}^{n}(x-x_{k}) 
\]

\ \ \ \ \ \ \ \ \ 

\end{document}
