
\documentclass{article}
%%%%%%%%%%%%%%%%%%%%%%%%%%%%%%%%%%%%%%%%%%%%%%%%%%%%%%%%%%%%%%%%%%%%%%%%%%%%%%%%%%%%%%%%%%%%%%%%%%%%%%%%%%%%%%%%%%%%%%%%%%%%%%%%%%%%%%%%%%%%%%%%%%%%%%%%%%%%%%%%%%%%%%%%%%%%%%%%%%%%%%%%%%%%%%%%%%%%%%%%%%%%%%%%%%%%%%%%%%%%%%%%%%%%%%%%%%%%%%%%%%%%%%%%%%%%
\usepackage{amsfonts}

%TCIDATA{OutputFilter=LATEX.DLL}
%TCIDATA{Version=5.50.0.2953}
%TCIDATA{<META NAME="SaveForMode" CONTENT="1">}
%TCIDATA{BibliographyScheme=Manual}
%TCIDATA{Created=Saturday, July 16, 2022 16:34:14}
%TCIDATA{LastRevised=Tuesday, July 26, 2022 09:45:55}
%TCIDATA{<META NAME="GraphicsSave" CONTENT="32">}
%TCIDATA{<META NAME="DocumentShell" CONTENT="Standard LaTeX\Blank - Standard LaTeX Article">}
%TCIDATA{CSTFile=40 LaTeX article.cst}

\newtheorem{theorem}{Theorem}
\newtheorem{acknowledgement}[theorem]{Acknowledgement}
\newtheorem{algorithm}[theorem]{Algorithm}
\newtheorem{axiom}[theorem]{Axiom}
\newtheorem{case}[theorem]{Case}
\newtheorem{claim}[theorem]{Claim}
\newtheorem{conclusion}[theorem]{Conclusion}
\newtheorem{condition}[theorem]{Condition}
\newtheorem{conjecture}[theorem]{Conjecture}
\newtheorem{corollary}[theorem]{Corollary}
\newtheorem{criterion}[theorem]{Criterion}
\newtheorem{definition}[theorem]{Definition}
\newtheorem{example}[theorem]{Example}
\newtheorem{exercise}[theorem]{Exercise}
\newtheorem{lemma}[theorem]{Lemma}
\newtheorem{notation}[theorem]{Notation}
\newtheorem{problem}[theorem]{Problem}
\newtheorem{proposition}[theorem]{Proposition}
\newtheorem{remark}[theorem]{Remark}
\newtheorem{solution}[theorem]{Solution}
\newtheorem{summary}[theorem]{Summary}
\newenvironment{proof}[1][Proof]{\noindent\textbf{#1.} }{\ \rule{0.5em}{0.5em}}
\input{tcilatex}
\begin{document}


\[
\text{\emph{Polinomios de Tchebyshev}} 
\]

\fbox{\emph{Definici\'{o}n:}}

Para cada $n$ $\in 
%TCIMACRO{\U{2115} }%
%BeginExpansion
\mathbb{N}
%EndExpansion
$,\textit{\ la funci\'{o}n }$T_{n}:\left[ -1,1\right] \rightarrow 
%TCIMACRO{\U{211d} }%
%BeginExpansion
\mathbb{R}
%EndExpansion
$ \textit{dada por }$T_{n}(x)=\cos (n$ $\arccos (x))$ \textit{se denomina n-%
\'{e}simo polinomio de Tchebyshev si }$n=0$, $T_{0}(x)=\cos (0)=1$

$n=1,$ $T_{1}(x)=x$

\textit{Sea }$\arccos (x)=\theta $,\textit{\ entonces tenemos lo siguiente:}

$T_{n+1}(n)=\cos ((n+1)\theta )=\cos (n\theta +\theta )=\cos (n\theta )\cos
(\theta )-\sin (n\theta )\sin (\theta )$

$T_{n-1}(n)=\cos ((n-1)\theta )=\cos ((n\theta -\theta ))=\cos (n\theta
)\cos (\theta )+\sin (n\theta )\sin (\theta )$

\textit{Sumando miembro a miembro se tiene }$T_{n+1}(x)+T_{n-1}(x)=2\cos
(\theta )\cos (n\theta )=2xT_{n}(x)$ \textit{donde: }\fbox{$%
T_{n+1}=2xT_{n}(x)-T_{n-1}(x)$}

\fbox{\emph{Teorema: }}

\textit{La familia }$A=\left\{ T_{k}\right\} _{k\geq 0}$ \textit{de
polinomios de Tchebyshev es una familia ortogonal en }$c(\left[ -1,1\right]
) $ \textit{el producto interno dado por }$(f|g)=\int%
\limits_{-1}^{1}f(x)g(x)\omega (x),$ \textit{de donde }$\omega (x)=\frac{1}{%
\sqrt[2]{1-x^{2}}}.$

\fbox{\emph{Demostraci\'{o}n:}}

\textit{Sean }$j,k\in 
%TCIMACRO{\U{2115} }%
%BeginExpansion
\mathbb{N}
%EndExpansion
$ \textit{con }$j\neq k$, \textit{queremos que }$(T_{j}|T_{k})=0$ $%
\rightarrow (T_{j}|T_{k})=\int\limits_{-1}^{1}T_{j}(x)T_{k}(x)\omega
(x)dx=\int\limits_{-1}^{1}\cos (j\arccos (x))\cos (\alpha \arccos (x))\frac{1%
}{\sqrt[2]{1-x^{2}}}dx.$

Sea $\theta =\arccos (x)$\textit{, entonces }$\partial \theta =\frac{-1}{%
\sqrt[2]{1-x^{2}}}dx$\textit{, de modo que }$(T_{j}|T_{k})=\int\limits_{0}^{%
\pi }\cos (j\theta )\cos (k\theta )d\theta =\left[ \frac{1}{k}\sin (k\theta
)\cos (j\theta )\right] _{0}^{\pi }+\frac{j}{k}\int\limits_{0}^{\pi }\sin
(j\theta )\sin (k\theta )d\theta $

\ \ \ \ \ \ \ \ \ \ \ \ \ \ \ \ \ \ \ \ \ \ \ \ \ \ \ \ \ \ \ \ \ \ \ \ \ \
\ \ \ \ \ \ \ \ \ \ \ \ \ \ \ \ \ \ \ \ \ \ \ \ \ \ \ \ \ \ \ \ \ \ \ \ \ \
\ \ \ \ \ \ \ \ \ \ \ \ \ \ \ \ \ \ \ $=\frac{j}{k}\left[ -\frac{1}{k}\left[
\sin (j\theta )\cos (k\theta )\right] _{0}^{\pi }+\frac{j}{k}%
\int\limits_{0}^{\pi }\cos (j\theta )\cos (k\theta )\right] $

\ \ \ \ \ \ \ \ \ \ \ \ \ \ \ \ \ \ \ \ \ \ \ \ \ \ \ \ \ \ \ \ \ \ \ \ \ \
\ \ \ \ \ \ \ \ \ \ \ \ \ \ \ \ \ \ \ \ \ \ \ \ \ \ \ \ \ \ \ \ \ \ \ \ \ \
\ \ \ \ \ \ \ \ \ \ \ \ \ \ \ \ \ \ \ $=\frac{j^{2}}{k^{2}}(T_{j}|T_{k}),$%
\textit{esto es }$(T_{j}|T_{k})=\frac{j^{2}}{k^{2}}$. \textit{Si }$%
(T_{i}|T_{k})\neq 0$

\textit{Tenemos que }$1=\frac{j^{2}}{k^{2}}$ \textit{contradiciendo que }$%
j\neq k$ \textit{por lo tanto tenemos que }$(T_{j}|T_{k})=0$\textit{\ si }$%
j\neq k$.

$\bullet $ \textit{Observaci\'{o}n: }$\left\Vert T_{0}\right\Vert
^{2}=(T_{0}|T_{0})=\int\limits_{-1}^{1}T_{k}(x)T_{k}(x)\omega
(x)dx=\int\limits_{-1}^{1}\cos ^{2}(x)(k\arccos (x))\frac{1}{\sqrt[2]{1-x^{2}%
}}dx=\int\limits_{0}^{\pi }\cos ^{2}(k\theta )d\theta =\int\limits_{0}^{\pi
}(\frac{1}{2}+\frac{1}{2}\cos (2k\theta ))d\theta =$\fbox{$\frac{\pi }{2}$}

\textit{Finalmente }$(f|T_{k})=\int\limits_{-1}^{1}f(x)T_{k}(x)\omega
(x)dx=\int\limits_{-1}^{1}f(x)\cos (k\arccos (x))=\frac{1}{\sqrt[2]{1-x^{2}}}%
dx=\int\limits_{0}^{\pi }f(\cos (\theta ))\cos (k\theta )d\theta $ \textit{%
luego si }$f\in $\textit{\ }$c(\left[ -1,1\right] )$ \textit{la mejor
aproximac\'{\i}\'{o}n usando los polinomios de Tchebyshev es }$%
g=\sum\limits_{k=0}^{m}c_{k}T_{k}$ \textit{donde }\fbox{$c_{0}=\frac{%
(f|T_{0})}{\pi }$} \textit{y }\fbox{$c_{k}=\frac{2}{\pi }(f|T_{k})$}.

\textit{Sea }$f\in c(\left[ a,b\right] )$ \textit{y sea }$g:\left[ a,b\right]
\rightarrow \left[ -1,1\right] $,\textit{\ dada por }$g(x)=\frac{2}{b-a}x-%
\frac{a+b}{b-a}$ \textit{entonces }$g\left( \left[ a,b\right] \right) =\left[
-1,1\right] $ \textit{y para cada }$k\in 
%TCIMACRO{\U{2115} }%
%BeginExpansion
\mathbb{N}
%EndExpansion
$,\textit{\ sea }$\widetilde{T}_{k}(x)=T_{k}(g(x))$.\textit{\ Sea ademas }$%
\widetilde{\omega }(x)=\omega (g(x))$ \textit{donde }$\omega (x)=\frac{1}{%
\sqrt[2]{1-x^{2}}}$.\textit{\ Veamos que la familia }$A=\left\{ \widetilde{T}%
_{k}\right\} _{k\geq 0}\subset c(\left[ a,b\right] )$ \textit{es una familia
ortogonal con el producto interno }$(f|g)=\int\limits_{a}^{b}f(x)g(x)\omega
(x)dx.$

$(\widetilde{T}_{k}|\widetilde{T}_{j})=\int\limits_{a}^{b}\widetilde{T}%
_{k}(x)\widetilde{T}_{j}(x)\widetilde{\omega }(x)dx=\int\limits_{a}^{b}%
\widetilde{T}_{k}(g(x))\widetilde{T}_{j}(g(x))\widetilde{\omega }(g(x))dx.$ 
\textit{Hacemos ahora el cambio de variable }$g(x)=t\,$,\textit{\ entonces }$%
dt=g^{\prime }(x)dx=\frac{2}{b-a}dx$,\textit{\ por lo tanto }$(\widetilde{T}%
_{k}|\widetilde{T}_{j})=\int\limits_{-1}^{1}T_{k}(t)T_{j}(t)\omega (t)\frac{%
b-a}{2}dt=\frac{b-a}{2}\int\limits_{-1}^{1}T_{k}(t)T_{j}(t)\omega (t)dt=0$ 
\textit{si }$j\neq k$.

$\left\Vert \widetilde{T}_{0}\right\Vert ^{2}=(\widetilde{T}_{0}|\widetilde{T%
}_{0})=\frac{b-a}{2}(T_{0}|T_{0})=\frac{\pi }{2}(b-a)$

$\left\Vert \widetilde{T}_{k}\right\Vert ^{2}=(\widetilde{T}_{k}|\widetilde{T%
}_{k})=\frac{b-a}{2}(\widetilde{T}_{k}|\widetilde{T}_{k})=\frac{\pi }{4}%
(b-a) $

$c_{0}=\frac{(f|\widetilde{T}_{0})}{\left\Vert \widetilde{T}_{0}\right\Vert
^{2}}=\frac{2}{\pi (b-a)}(f|\widetilde{T}_{0})$

$c_{k}=\frac{4}{\pi (b-a)}(f|\widetilde{T}_{k})$ \textit{si }$k\geq 1.$

$(f|\widetilde{T}_{k})=\int\limits_{a}^{b}f(x)\widetilde{T}_{k}(x)\widetilde{%
\omega }(x)dx=\int\limits_{a}^{b}f(x)T_{k}(g(x))\omega (x)dx=$

\ \ \ \ \ \ \ \ $=\int\limits_{a}^{b}f(x)(k\arccos (\theta ))\frac{1}{\sqrt[2%
]{1-g^{2}(x)}}dx=\int\limits_{0}^{\pi }f(g^{-1}(\cos (\theta )))\cos
(k\theta )\frac{b-a}{2}d\theta =$\frame{$\frac{b-a}{2}\int\limits_{0}^{\pi
}fg^{-1}(\cos (\theta ))\cos (k\theta )d\theta $}.

$c_{k}=\frac{4}{\pi (b-a)}\frac{b-a}{2}\int\limits_{0}^{\pi }f(g^{-1}(\cos
(\theta )))\cos (k\theta )d\theta =\frac{2}{\pi }\int\limits_{0}^{\pi
}f(g^{-1}(\cos (\theta )))\cos (k\theta )d\theta =$\frame{$%
\int\limits_{0}^{\pi }f(\frac{2\cos (\theta )}{b-a}+\frac{2(b+a)}{(b-a)^{2}}%
)\cos (k\theta )d\theta $}

\end{document}
