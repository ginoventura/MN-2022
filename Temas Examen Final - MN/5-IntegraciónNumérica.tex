
\documentclass{article}
%%%%%%%%%%%%%%%%%%%%%%%%%%%%%%%%%%%%%%%%%%%%%%%%%%%%%%%%%%%%%%%%%%%%%%%%%%%%%%%%%%%%%%%%%%%%%%%%%%%%%%%%%%%%%%%%%%%%%%%%%%%%%%%%%%%%%%%%%%%%%%%%%%%%%%%%%%%%%%%%%%%%%%%%%%%%%%%%%%%%%%%%%%%%%%%%%%%%%%%%%%%%%%%%%%%%%%%%%%%%%%%%%%%%%%%%%%%%%%%%%%%%%%%%%%%%
%TCIDATA{OutputFilter=LATEX.DLL}
%TCIDATA{Version=5.50.0.2953}
%TCIDATA{<META NAME="SaveForMode" CONTENT="1">}
%TCIDATA{BibliographyScheme=Manual}
%TCIDATA{Created=Wednesday, July 13, 2022 19:36:15}
%TCIDATA{LastRevised=Tuesday, July 26, 2022 09:43:43}
%TCIDATA{<META NAME="GraphicsSave" CONTENT="32">}
%TCIDATA{<META NAME="DocumentShell" CONTENT="Standard LaTeX\Blank - Standard LaTeX Article">}
%TCIDATA{CSTFile=40 LaTeX article.cst}

\newtheorem{theorem}{Theorem}
\newtheorem{acknowledgement}[theorem]{Acknowledgement}
\newtheorem{algorithm}[theorem]{Algorithm}
\newtheorem{axiom}[theorem]{Axiom}
\newtheorem{case}[theorem]{Case}
\newtheorem{claim}[theorem]{Claim}
\newtheorem{conclusion}[theorem]{Conclusion}
\newtheorem{condition}[theorem]{Condition}
\newtheorem{conjecture}[theorem]{Conjecture}
\newtheorem{corollary}[theorem]{Corollary}
\newtheorem{criterion}[theorem]{Criterion}
\newtheorem{definition}[theorem]{Definition}
\newtheorem{example}[theorem]{Example}
\newtheorem{exercise}[theorem]{Exercise}
\newtheorem{lemma}[theorem]{Lemma}
\newtheorem{notation}[theorem]{Notation}
\newtheorem{problem}[theorem]{Problem}
\newtheorem{proposition}[theorem]{Proposition}
\newtheorem{remark}[theorem]{Remark}
\newtheorem{solution}[theorem]{Solution}
\newtheorem{summary}[theorem]{Summary}
\newenvironment{proof}[1][Proof]{\noindent\textbf{#1.} }{\ \rule{0.5em}{0.5em}}
\input{tcilatex}
\begin{document}


\[
\text{\emph{Intregraci\'{o}n Num\'{e}rica}\textit{\ }} 
\]

\textit{Sea }$f:\left[ a,b\right] \rightarrow R$\textit{\ continua y sea }$%
P=\left\{ x_{1}/a=x_{0}<x_{1}<x_{2}<\cdots <x_{n}=b\right\} $\textit{\ una
partici\'{o}n del intervalo }$\left[ a,b\right] $ \textit{tal que }$%
x_{k}=x_{0}+kh$\textit{, donde }\fbox{$h=\frac{b-a}{n}$}.\textit{\
Aproximaremos el valor de }$\int_{a}^{b}f(x)dx$ \textit{con }$%
I(f)=\int_{a}^{b}L(x)dx$\textit{\ donde }\fbox{\textit{\ }$L$\textit{\ es el
polinomio interpolante de Lagrange }} \textit{para el conjunto de puntos }$%
S=\left\{ (x_{i},f(x_{i}))\right\} _{i=0}^{n}$. \textit{Por lo tanto }\fbox{$%
L(x)=\sum\limits_{i=0}^{n}f(x_{i})L_{i}(x)$} , donde \fbox{$L_{i}(x)=\frac{%
\prod\limits_{j\neq i}(x-x_{j})}{\prod\limits_{j\neq i}(x_{i}-x_{j})}$}

\textit{Entonces, }$I_{n}(f)=\int_{a}^{b}L(x)dx=\int_{a}^{b}\left[
\sum\limits_{i=0}^{n}f(x_{i})L_{i}(x)\right] dx=\sum%
\limits_{i=0}^{n}f(x_{i})\int_{a}^{b}L_{i}(x)dx=\sum%
\limits_{i=0}^{n}f(x_{i})c_{i}$,\textit{\ donde, }$c_{i}=%
\int_{a}^{b}L_{i}(x)dx$.

\textit{Miremos }$c_{i}$\textit{\ en detalle:}

$c_{i}=\frac{1}{\prod\limits_{j\neq i}(x_{i}-x_{j})}\dint_{a}^{b}\left[
\prod\limits_{j\neq i}(x-x_{j})\right] dx=$

$\frac{1}{\prod\limits_{j\neq i}(x_{0}+ih-x_{0}-jh)}\dint_{a}^{b}\left[
\prod\limits_{j\neq i}(x-x_{j})\right] dx=$

$\frac{1}{\prod\limits_{j\neq i}(i-j)h}\dint_{a}^{b}\left[
\prod\limits_{j\neq i}(x-x_{j})\right] dx=$

\fbox{$\frac{1}{h^{n}\prod\limits_{j\neq i}(i-j)}\dint_{a}^{b}\left[
\prod\limits_{j\neq i}(x-x_{j})\right] dx$}.\textit{\ }

\textit{Hacemos ahora el cambio de variable }$x=x_{0}+th$\textit{, entonces }%
$h=dt$\textit{\ y por lo tanto}

$c_{i}=$\textit{\ }$\frac{1}{h^{n}\prod\limits_{j\neq i}(i-j)}\dint_{0}^{n}%
\left[ \prod\limits_{j\neq i}(x_{0}+th-x_{0}-jh)\right] h$ $dt=$

$\frac{1}{h^{n}\prod\limits_{j\neq i}(i-j)}\dint_{0}^{n}\left[
\prod\limits_{j\neq i}(t-j)h\right] dt=$

$\frac{h^{n+1}}{h^{n}\prod\limits_{j\neq i}(i-j)}\dint_{0}^{n}\left[
\prod\limits_{j\neq i}(t-j)h\right] dt=$

\fbox{$\frac{h}{\prod\limits_{j\neq i}(i-j)}\dint_{0}^{n}\left[
\prod\limits_{j\neq i}(t-j)h\right] dt$}

\fbox{\emph{Ejemplo}}

\fbox{$n=1$}

$I(f)=c_{0}(f(x_{0}))+c_{1}(f(x_{1}))$\ \ $=$\ \fbox{$\frac{x_{1}-x_{0}}{2}%
\left[ f(x_{0})+f(x_{1})\right] \rightarrow \frac{b-a}{2}=(f(a)+f(b))$}

\fbox{$n=2$}

\fbox{$I_{2}(f)=\frac{h}{3}\left[ f(x_{0})+4f(x_{0}+h)+f(x_{0}+2h)\right]
\rightarrow $ \emph{Metodo de Simpson}}

\[
\text{\emph{Error en la intregraci\'{o}n Num\'{e}rica}} 
\]

\textit{Si }$R_{n}=I(f)-\dint_{a}^{b}f(x)$ $dx$, \textit{entonces }$R_{n}(f)$
\textit{est\'{a} dado por:}\ \ \ \ \ \ \ \ \ \ \ \ \ \ \ \ \ \ \ \ \ \ \ \ 

\ \ \ \ \ \ \ \ \ \ \ \ \ \ \ \ \ \ \ \ \ \ \ \ \ \ \ \ \ \ \ \ \ \ \ \ \ \
\ $R_{n}(f)=%
\begin{array}{c}
\frac{R_{n}(x^{n+1})}{(n+1)!}f^{(n+1)}(\xi )\rightarrow \text{\textit{\ Si }}%
n\text{\textit{\ es impar}} \\ 
\frac{R_{n}(x^{n+2})}{(x+2)!}f^{(n+2)}(\xi )\rightarrow \text{ \textit{Si }}n%
\text{\textit{\ es par}}%
\end{array}%
$ $\ \ \ \ \xi \in \left( a,b\right) $

\fbox{\textit{Para }$n=1$}

\fbox{$R_{1}(f)=\frac{h^{3}}{12}f^{\prime \prime }(\xi )$}

\end{document}
