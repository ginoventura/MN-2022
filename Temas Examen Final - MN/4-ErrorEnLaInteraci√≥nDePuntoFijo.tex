
\documentclass{article}
%%%%%%%%%%%%%%%%%%%%%%%%%%%%%%%%%%%%%%%%%%%%%%%%%%%%%%%%%%%%%%%%%%%%%%%%%%%%%%%%%%%%%%%%%%%%%%%%%%%%%%%%%%%%%%%%%%%%%%%%%%%%%%%%%%%%%%%%%%%%%%%%%%%%%%%%%%%%%%%%%%%%%%%%%%%%%%%%%%%%%%%%%%%%%%%%%%%%%%%%%%%%%%%%%%%%%%%%%%%%%%%%%%%%%%%%%%%%%%%%%%%%%%%%%%%%
\usepackage{amsmath}

\setcounter{MaxMatrixCols}{10}
%TCIDATA{OutputFilter=LATEX.DLL}
%TCIDATA{Version=5.50.0.2953}
%TCIDATA{<META NAME="SaveForMode" CONTENT="1">}
%TCIDATA{BibliographyScheme=Manual}
%TCIDATA{Created=Wednesday, July 13, 2022 18:52:20}
%TCIDATA{LastRevised=Tuesday, July 26, 2022 09:42:21}
%TCIDATA{<META NAME="GraphicsSave" CONTENT="32">}
%TCIDATA{<META NAME="DocumentShell" CONTENT="Standard LaTeX\Blank - Standard LaTeX Article">}
%TCIDATA{CSTFile=40 LaTeX article.cst}

\newtheorem{theorem}{Theorem}
\newtheorem{acknowledgement}[theorem]{Acknowledgement}
\newtheorem{algorithm}[theorem]{Algorithm}
\newtheorem{axiom}[theorem]{Axiom}
\newtheorem{case}[theorem]{Case}
\newtheorem{claim}[theorem]{Claim}
\newtheorem{conclusion}[theorem]{Conclusion}
\newtheorem{condition}[theorem]{Condition}
\newtheorem{conjecture}[theorem]{Conjecture}
\newtheorem{corollary}[theorem]{Corollary}
\newtheorem{criterion}[theorem]{Criterion}
\newtheorem{definition}[theorem]{Definition}
\newtheorem{example}[theorem]{Example}
\newtheorem{exercise}[theorem]{Exercise}
\newtheorem{lemma}[theorem]{Lemma}
\newtheorem{notation}[theorem]{Notation}
\newtheorem{problem}[theorem]{Problem}
\newtheorem{proposition}[theorem]{Proposition}
\newtheorem{remark}[theorem]{Remark}
\newtheorem{solution}[theorem]{Solution}
\newtheorem{summary}[theorem]{Summary}
\newenvironment{proof}[1][Proof]{\noindent\textbf{#1.} }{\ \rule{0.5em}{0.5em}}
\input{tcilatex}
\begin{document}


\begin{equation*}
\text{\emph{Error en la iteraci\'{o}n de Punto Fijo}}
\end{equation*}

\textit{Sea }$g:R\rightarrow R$ \textit{de clase }$C^{1}$ \textit{con }$%
\left\vert g^{\prime }(x)\right\vert \leq r<1$ \textit{y sea la sucesi\'{o}n 
}$\left\{ P_{n}\right\} $\textit{\ dada por }$P_{n}=g(P_{n-1})(n\geq 1)$ 
\textit{tal que }$\underset{n\rightarrow \infty }{\lim }P_{n}=P(g(P)=P)$. 
\textit{Si }$n,m$ $\in Z^{+}$ \textit{con }$m>n,\left\vert
P_{m}-P_{n}\right\vert \leq \frac{r^{n}-r^{m}}{1-r}\left\vert
P_{1}-P_{0}\right\vert $ \textit{y tomamos limite para }$m\rightarrow \infty 
$ \textit{tenemos }$\left\vert P-P_{n}\right\vert \leq \frac{r^{n}}{1-r}%
\left\vert P_{1}-P_{0}\right\vert .$

\textit{Sea }$f:R\rightarrow R$ \textit{de clase }$C^{1}.$ \textit{Queremos
hallar (si existen) puntos }$x_{0}\in dom(f)$ \textit{tales que }$%
f(x_{0})=0. $ \textit{Por razones pr\'{a}cticas, escribiremos los elementos
de }$R^{n}$ \textit{como vectores columna. Aplicando el m\'{e}todo de
Newton-Raphson tenemos }$G:R^{n}\rightarrow R^{n}$ \textit{dada por }$G(x)=x-%
\left[ (JF)(x)\right] ^{-1}f(x)$, \textit{entonces la sucesi\'{o}n
(vectorial) }$\left\{ P_{n}\right\} \mathit{\ }$\textit{dada por }$%
P_{n}=G(P_{n-1})$\textit{\ tiende a un punto fijo }$P$ \textit{tal que }$%
f(P)=0.$

\fbox{\emph{Ejemplo}}

$f(x,y)=(xe^{xy}(2+xy),x^{3}e^{xy}-1)$

\textit{Matriz Jacobiana:}

$\ $\textit{forma }$\rightarrow \bigskip \left[ 
\begin{array}{cc}
\partial xx & \partial yx \\ 
\partial xy & \partial yy%
\end{array}%
\right] \rightarrow JF(x,y)=\left[ 
\begin{array}{cc}
(2+4xy+x^{2}y^{2})e^{xy} & e^{xy}x^{2}(3+xy) \\ 
(3x^{2}+x^{3}y)e^{xy} & x^{4}e^{xy}%
\end{array}%
\right] $

$G(x,y)=\left[ 
\begin{array}{c}
x \\ 
y%
\end{array}%
\right] -\left[ 
\begin{array}{cc}
(2+4xy+x^{2}y^{2})e^{xy} & e^{xy}x^{2}(3+xy) \\ 
(3x^{2}+x^{3}y)e^{xy} & x^{4}e^{xy}%
\end{array}%
\right] \left[ 
\begin{array}{c}
xe^{xy}(2+xy) \\ 
x^{3}e^{xy}-1%
\end{array}%
\right] \leftarrow 
\begin{array}{c}
\partial x \\ 
\partial y%
\end{array}%
$

\end{document}
